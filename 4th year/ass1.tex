\documentclass[12pt, letterpaper]{article}
\usepackage{bbold}
\usepackage{indentfirst}
\usepackage{amsmath, amssymb}
\usepackage[T1]{fontenc}
\usepackage[utf8]{inputenc}
\usepackage{physics}
\usepackage{tensor}
\usepackage{braket}
\usepackage{graphics}
\usepackage{grffile}

\begin{document}
    Consider a sample of radioactive nuclei at time . Also consider that is the number
    of particles that decay in some time interval .
    From discrete decay model we can write
    \begin{equation}
        \frac{\Delta N(t)}{\Delta t}  = -\lambda N(t)
    \end{equation}
    
    From continuous decay model (for $N \to \infty$  and $t \to 0$ )

    \begin{equation}
        \frac{dN(t)}{dt}  = -\lambda N(t)
    \end{equation}

    (a) Plot the logarithm of the number of radioactive particle left $[\ln N(t)]$ versus time and
    logarithm of the decay rate $[\ln(\Delta N(t)) ]$ versus time. You will obtain exponential decay
    when you start with large values of and a stochastic process for small N(0).

    (b) Create two plots, one showing that the slopes of plots of N(t) versus t are independent
    of N(0) and the another showing that the slope is proportional to $\lambda$.
\end{document}