\documentclass[12pt, letterpaper]{article}
\usepackage{bbold}
\usepackage{indentfirst}
\usepackage{amsmath, amssymb}
\usepackage[T1]{fontenc}
\usepackage[utf8]{inputenc}
\usepackage{physics}
\usepackage{tensor}
\usepackage{braket}
\usepackage{graphics}
\usepackage{grffile}
\usepackage[export]{adjustbox}
\usepackage{svg}
\usepackage{caption}
\usepackage{subcaption}
\usepackage{authblk}
\usepackage{setspace}
\newcommand*{\1}{\hspace{1pt}}

\title {General Relativity and Cosmology}

\author{}

\date{28 May, 2022}


\begin{document}
    \maketitle

    \subsection*{Very Quick Recap of Relativity}
    
    Everywhere we see are events. Spacetime is a collection of events. We can refer events as a stage drama. Coordinates are used to described these events. 
    Coordinates are local.  

    \subsection*{Manifolds}
    Manifold is a set endowed with a collection of subsets which 
    But according to physicists, where we can do integration and differentiations on, are called manifolds.

    \subsection*{World lines}
    World lines $\longleftrightarrow$ observables.
    
    squared distance ds will be 

    \begin{equation}
        ds^{2} = -dt^{2} + d\underline{x} \cdot d\underline{x}  
    \end{equation}

    Signature of our spacetime will be (-, +, +, +) which is known as East Coast Convention. Where

    \begin{align*} 
            A \cdot A & > 0 \tag*{spacelike} \\ 
            A \cdot A & < 0 \tag*{timelike} \\ 
            A \cdot A & = 0 \tag*{null or lightlike} 
    \end{align*}

    \subsection*{Proper Time}
    Time measured by the observer is called proper time. \\

    Now we define scalar and vectors 

    \subsection*{Scalar}
    Quantities that are invariant under coordinate transformation are called scalars.

    \begin{equation*}
        P(x) = P(x^{'})
    \end{equation*}

    \subsection*{Vector}

    Collection of tangent lines on a point at a time direct to everywhere are called tangent vector.
    
    $T_{p}$ is a vector space. If 

    \begin{equation}
        \begin{split}
            U \in T_{p} \\
            V \in T_{p} \\ 
            \\ 
            \alpha U + \beta V \in T_{p}
        \end{split}
    \end{equation}

    \begin{align*}
        \frac{df}{d \tau} = \frac{\partial f}{\partial x^{i}} \frac{\partial x^{i}}{\partial \tau} = (V^{i}(x) \frac{\partial}{\partial x_{i}})f \tag*{ $[dx^{i} = V^{i}]$} \\   
    \end{align*}

    Functionally, 
    \begin{equation}
        \frac{d}{d \tau} \equiv 
         (V^{i}(x) \frac{\partial}{\partial x_{i}})
    \end{equation}

    here $\frac{\partial}{\partial x^{i}}$ is coordinate basis. However, the coordinates, $x^{i}$ are observer's contruct. The coordinate basis is independent. 

    \begin{equation}
        \left[ \frac{\partial}{\partial x^{i}}, \frac{\partial}{\partial x^{j}}\right] = 0 
    \end{equation}

    $\frac{d}{d \tau}$ causes a translation along the tangent of the curve. It is independent of the coordinates x's.

    \subsection*{Lie Bracket}
    \begin{equation}
        \hat{V} \equiv  \frac{d}{d \tau} \equiv  (V^{i}(x) \frac{\partial}{\partial x_{i}})
    \end{equation}

    \begin{equation}
        \begin{split}
        \left[\hat{V}, \hat{W}\right] & = \hat{V}\hat{W} - \hat{W}\hat{V}  \\ 
            & = \left(V^{j}\frac{\partial}{\partial x^{j}}\right)\left(W^{i}\frac{\partial}{\partial x^{i}}\right) - \left(W^{j}\frac{\partial}{\partial x^{j}}\right)\left(V^{i}\frac{\partial}{\partial x^{i}}\right)   \\
            & = \left(V^{i}\frac{\partial W^{j}}{\partial x^{i}} - W^{i}\frac{\partial V^{j}}{\partial x^{i}}\right)\frac{\partial}{\partial x^{j}} \\ 
            & = T^{j}\frac{\partial}{\partial x^{j}} 
        \end{split}
    \end{equation}

    Where 
    \begin{equation*}
        T^{j} = \left[\hat{V}, \hat{W}\right]^{j} =  \left(V^{i}\frac{\partial W^{j}}{\partial x^{i}} - W^{i}\frac{\partial V^{j}}{\partial x^{i}}\right)\frac{\partial}{\partial x^{j}}
    \end{equation*}

    \begin{equation}
        \hat{V} = \frac{d}{d \tau} = (V^{i}(x) \frac{\partial}{\partial x_{i}}) = V^{'a}({x^{'}})\frac{\partial}{\partial x^{'a}}
    \end{equation}

    \begin{equation*}
        \begin{split}
            x & = x(x^{'}) \\ 
            x^{'} & = x^{'}(x) \\
            \frac{\partial ?}{\partial x^{i}} & = \frac{\partial x^{'a}}{\partial x^{i}}\frac{\partial ?}{\partial x^{'a}} 
        \end{split}
    \end{equation*}

    \begin{equation}
        \begin{split}
        \hat{V} = V^{'a}({x^{'}})\frac{\partial}{\partial x^{'a}} & = \left(V^{i}({x})\frac{\partial x^{'a}}{\partial x^{i}}\right)\frac{\partial}{\partial x^{'a}} \\ 
        \implies V^{'a}{x^{'}}  & = V^{i}(x)\frac{\partial x^{'a}}{\partial x^{i}}
        \end{split}
    \end{equation}

    Diffumorphism maps manifold to itself.
\end{document}