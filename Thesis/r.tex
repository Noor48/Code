\documentclass[12pt, letterpaper]{article}
\usepackage{bbold}
\usepackage{indentfirst}
\usepackage{amsmath, amssymb}
\usepackage[T1]{fontenc}
\usepackage[utf8]{inputenc}
\usepackage{physics}
\usepackage{tensor}
\usepackage{braket}
\usepackage{graphics}
\usepackage{grffile}
\usepackage[export]{adjustbox}
\usepackage{svg}
\usepackage{caption}
\usepackage{subcaption}
\usepackage{authblk}
\usepackage{setspace}

\begin{document}
    \section*{AIMD Simulation Method}
    AIMD simulations that make use of the Vienna ab initio simulation package's implementation of the density functional theory based on plane waves. It was cut 
    off at 400 eV for the plane-wave basis set. The projected augmented-wave approach was used to represent the electron-ion interaction.
    These simulation method is divided into 2 steps. Fisrt one is Standard relaxation. And second one is Molecular dynamics. \\ 
     \\ 
    For Standard relaxation we prepare the cell of a selected material using VESTA(visualization for electronic structural analysis)
    The Nose-Hoover Thermostat
    was used in the NVT ensemble to regulate temperature during all of the dynamic simulations. Verlet's approach was used to integrate Newton's equation of motion 
    in the velocity domain with a time step of 3 fs and a total of 600 time steps. In simulations, every material was employed in a cubic cell containing 100 atoms. 
    The supercell Brillioun zone was thought to be sampled via gamma point sampling. For a seamless refining of the structure, we first loosened the structures. To 
    attain thermal equilibrium, the liquid samples of the chosen materials were first prepared at 2000K, which is much higher than their melting points. 200 steps of
    MD were user to extract quantities for pair correlation function.
\end{document}