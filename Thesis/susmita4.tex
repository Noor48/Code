\documentclass[12pt]{article}
\usepackage{amsmath}
\usepackage{color}
\usepackage[utf8]{inputenc}
\usepackage{breqn}
\usepackage{physics}
\usepackage{graphicx}
\usepackage[export]{adjustbox}
\usepackage{rotating}
\usepackage{caption}
\usepackage{changepage}
\usepackage{array,multirow,graphicx}
\newcommand*{\1}{\hspace{1pt}}
\graphicspath{/home/susmita/Documents/project/amar project}

\title{\color{blue} \Huge{ SEGREGATION OF BINARY ALLOYS}}

\author{ COURSE TITLE : HONOURS PROJECT \\ COURSE CODE : TP-407 \\ [20PT] \textbf{SUBMITTED BY}
 \\ [5PT] \Large SUSMITA SARKER \\ \Large CLASS ROLL : SN-098-032 \\ \Large EXAM ROLL: 30219 \\
  \Large REGISTRATION NUMBER : 2016-014-718 \\ \Large  ACADEMIC YEAR :  2019-20 \\
   \Large  SESSION : 2016-17
\\ [10pt] 
\includegraphics[width=0.5\textheight]{DU.jpg} \\ [10pt] \Large DEPARTMENT OF THEORETICAL PHYSICS \\ UNIVERSITY OF DHAKA 
}


\date{March 24,2022}

\begin{document}
	%\maketitle
	
	
\pagebreak

\rule{\textwidth}{0.5pt}
\begin{abstract}
Content
\end{abstract}

\rule{\textwidth}{0.5pt}
\newpage

    \section*{1. Introduction}
    Theoreticians have briefed the significance of the mixing properties of binary liquid alloys 
    from both the scientific and the technological points of view. A precise understanding of 
    the mixing properties and phase diagrams of the alloy system is elementary to establish a 
    good arrangement between the experimental results, theoretical approaches, and empirical 
    models for liquid alloys with a miscibility gap.
    All liquid binary alloys can be grouped into two distinct classes that either exhibit 
    positive deviation(usually called segregating systems) or negative deviation (i.e. 
    short-ranged ordered alloys) from Raoult's law or the additive rule of mixing. If the 
    deviations are considerably large, they may conduct either phase separation or compound 
    formation in the binary system. \\

        There are liquid alloys, however, which do not belong exclusively to any of the above 
    two classes. For instance, the excess Gibbs energy of mixing $(G ^{xs} _{M} )$ for Cd–Na, and 
    Ag–Ge is negative at certain compositions, while positive at other compositions. In 
    liquid alloys such as Au–Bi, Bi–Cd, and Bi–Sb, the enthalpy of mixing $H_{M}$ is a 
    positive quantity but $G^{xs} _{M}$  is negative. Bi–Pb has positive $H_{M}$ and $G^{xs}_{M} $ in the 
    solid phase as against the negative values of $H_{M}$ and $G^{xs} _{M}$ in the liquid phase. 
    Systems such as Au–Ni and Cr–Mo exhibit immiscibility in the solid phase which is not 
    visible in the interrelated liquid phase. Systems such as Ag–Te show intermetallic 
    phases and large negative $H_{M}$ values in the liquid phase concurrently with a liquid 
    miscibility gap. \\

    
        Systems such as Al–Bi, Al–In, Al–Pb, Bi–Ga, Bi–Zn, Cd–Ga, Ga–Pb, Ga–Hg, Pb–Zn, Pb–Si 
    and Cu–Pb, etc, are represented by liquid miscibility gaps and exhibit enormous positive 
    $H_{M}$. Their properties in the liquid phase tend to vary markedly as a role of composition (c),
    temperature (T ), and pressure (p). The long-wavelength limit (q→0)of the 
    composition–composition structure factors, $S_{cc}$ (0) diverges as the composition and 
    temperature approach the critical values c → $c_{c}$, and T → $T_{c}$. $S_{cc}$ (0), which can instantly
    be obtained from thermodynamic functions (either acquired by taking the first composition 
    derivative of the activity or through the second derivative of the Gibbs function), is 
    very useful for establishing the immiscibility and the degree of segregation in binary 
    liquid alloys. Additional thermodynamic, structural, and transport properties are also 
    found to alter anomalously in the area of $c_{c}$ and $T_{c}$. \\ 


        A provided unary system is expressed by two pairs of independent variables, namely 
    the mechanical degrees of freedom (pressure (p) or volume ($\Omega$)) and the thermal 
    degrees of freedom (temperature (T) or entropy (S)). The preference of independent 
    variables is mostly a concern of free option; yet, there are four possibilities for 
    creating such pairs have one mechanical and one thermal variable, say (S,$\Omega$ ), 
    (S, p), (T, $\Omega$) and (T, p). These pairs guide to thermodynamic functions such as 
    internal energy E(S,$\Omega$), enthalpy H (S, p), Helmholtz energy F (T, $\Omega$), and 
    Gibbs energy G(T, p), respectively.\\

    The enthalpy, H, merging the internal energy E to the mechanical degrees of freedom
    (p, $\Omega$) is\\
                    
                 \begin{equation}
                    H = E + p\Omega
                \end{equation}

    or in differential form,
                
                \begin{equation}
                    dH = \delta Q + \Omega dp
                \end{equation}

    where       \begin{equation*}
                    \delta Q = dE + pd \Omega
                \end{equation*}

    The Helmholtz energy, F, relates E to the thermal degrees of freedom (S,T), i.e.\\

                \begin{equation}
                    F = E - TS
                \end{equation}

    or,

                \begin{equation}
                    dF = -SdT - pd \Omega
                \end{equation}

    In the case of reversible isothermal and isochoric processes (T , $\Omega$ = constant),
    dF = 0, i.e.
    F remains invariant.
        Similarly, the Gibbs function establishes a relation between H and the thermal degrees
    of freedom, i.e.\\

                \begin{equation}
                G = H - TS
                \end{equation}

or

                \begin{equation}
                dG = -S dT + \Omega dp
                \end{equation}

    In the case of a reversible isothermal reaction at constant pressure (T , p = constant),
    dG = 0, i.e. G remains invariant.\\

        Also, H , F and G can readily be used to obtain the heat capacity C ($C_{p}$ or $C_{\Omega}$ ),
    entropy, isothermal ($\chi _{T}$ ) and adiabatic ($\chi _{S}$) compressibilities, the volume and the volume
    expansivity ($\alpha  _{p}$):\\

    \begin{align}
    &  C_{p} = \left(\frac{\partial H }{ \partial T}\right)_{p} = T \left(\frac{\partial S}{\partial T} \right) _{p} = -T \left(\frac{\partial ^2 G }{ \partial T^2}\right) _p \\
    & C_{\Omega} = \left(\frac{\partial E }{ \partial T}\right)_{\Omega} = T \left(\frac{\partial S}{\partial T} \right) _{\Omega} = -T \left(\frac{\partial ^2 F }{ \partial T^2}\right) _\Omega \\
    & S = \left(\frac{\partial G }{ \partial T}\right)_{p} =  \left(\frac{\partial F}{\partial T} \right) _{\Omega} \\
    & \Omega  = \left(\frac{\partial G }{ \partial p}\right)_{T} \\
    & \chi _T  \equiv  - \frac{1}{ \Omega} \left(\frac{\partial  \Omega }{ \partial p}\right)_{T} \\ 
    & \chi _S  \equiv  - \frac{1}{ \Omega} \left(\frac{\partial  \Omega }{ \partial p}\right)_{S} \\
    & \alpha  _p  \equiv  \frac{1}{ \Omega} \left(\frac{\partial  \Omega }{ \partial T}\right)_{p}  
    \end{align}\\

    At indicator, we furthermore have some important effects from isotherms of liquid–vapor phases which at the critical point must fulfill
    \begin{align}
    & \left(\frac{\partial p}{\partial \Omega} \right) _{T_{c}} = \left(\frac{\partial ^2 p}{\partial \Omega ^2} \right) _{T_{c}} =0
    \end{align}\\

    At T = $T_c$ , the following physical properties become infinite, i.e.\\


    \begin{align}
    & C_p = T \left(\frac{\partial S}{\partial T} \right) _p = \infty \\
    & \alpha _p = \frac{1}{\Omega} \left(\frac{\partial \Omega}{\partial T} \right) _p = \infty \\
    & \chi _T = -\left(\frac{\partial \Omega}{\partial p} \right) _T = \infty 
    \end{align}\\

    For a binary mixture, such as an A–B alloy consisting of $c_A$ moles of component A and $c_B$
    moles of component B, rather than guiding to the fundamental values of their function, 
    we define the function of mixing. For example, the Gibbs energy of mixing, $G_M$ , is represented as

    \begin{equation}
        G_M = G(alloy) - c_A G^{0}_{A} - c_B G^{0}_{B}
    \end{equation}

where $G ^{0}_{A}$ and $G ^{0}_{B}$ are the Gibbs free energy of the two pure components. Equivalent
definitions also exist for HM , SM and other functions.The integral quantities can also be divided 
into the partial quantities, i.e.

    \begin{equation}
        G_M = c_A \overline{G} _{M,A} + c_B \overline{G} _{M,B}
    \end{equation}
with
    \begin{equation}
        \overline{G} _{M,i} = RT ln a_i                \tag*{( i= A,B )}
    \end{equation}
where  $\overline{G} _{M,i} $ are the partial Gibbs energies and $a _{i}$ are the thermodynamic 
activities of the component i.
$G _M$ defines the stability of the phases in a binary mixture. The curves describing
$G _M$ against c deviation can, in general, have a shape like either curve a or curve b as displayed
in figure 1. For $G _M$ as in curve a, the homogeneous solution is stable at all values of c at
$T _1$ ; if not other phases (i.e. intermediate phases) in the system display more negative $G _M$
values.


    
\begin{adjustbox}{center,caption={A schematic diagram symbolizing the Gibbs energy of mixing at constant T plotted against concentration. 
    Curve a, complete mixing ($ T _1 < T _c $). 
    Curve b, incomplete mixing ($ T _2 < T _c $), 
    $\Delta c$ represents the miscibility gap at $ T _2$ .},label={somelabel},nofloat=figure,vspace=\bigskipamount}
    %\includegraphics[width=0.7\textwidth]{fi 1}
\end{adjustbox}
    

    For curve b, the homogeneous solution is varying in the composition range $\Delta c$, because
$G _M$ can be reduced if the mixture separates into two phases. The composition of these phases
is provided by the points of contact P and Q of the common tangent line to the $G _M$(c) curve.
The reduced $G _M$  values of these two phases are given by this line. Within the composition
range $\Delta c$ only the portions of the two phases change if the total composition of the alloy
modifications. At P($c _1$) and Q($c _2$) the partial Gibbs energies of the components of both diverged
phases are equivalent,

    \begin{equation}
        \overline{G} _{M,i}(c _{1}) =  \overline{G} _{M,i}(c _{2})          \tag*{( i= A,B )}
    \end{equation}

    Hence P and Q indicate the limit of thermodynamic equilibrium. $G _M$ diverts as a function of
temperature from a concave to a convex surface for $\Delta c $ at the spinodal. The points of
inflection in the curves define the spinodal line. The critical composition and the critical
temperature are determined from the conditions at T = $T _c$\\

    \begin{align}
        \left(\frac{d^2 {G _M}}{d c^2} \right) _{c = c_c} = 0 \\
        \left(\frac{d^3 {G _M}}{d c^3} \right) _{c = c_c} = 0 
    \end{align}


    At this step, it should be pointed out that the long-wavelength limit (q → 0) of
the structure factor $S _{cc} $(q) which is well known as the
concentration fluctuation, $S _{cc} $(0), is also correlated to the thermodynamic function, i.e

    \begin{equation}
        S_{cc}(0) =  RT \left(\frac{d^2 G _M}{d c^2} \right) ^ {-1}_ {T,P}
    \end{equation}
As $ c\to c _c$ and $ T\to T_c $ ,one sees that

    \begin{equation}
        S_{cc}(0) \to \infty
    \end{equation}  
Thus, a phase separation in a binary mixture is signaled by a strong enhancement of the
concentration fluctuations. 
The ideal solution behavior (HM = 0) of a binary mixture is presented by
    \begin{equation}
        G ^ {id} _M = RT (c_A \ ln c_A + c_B \ ln c_B).
    \end{equation}

The distinctions in the thermodynamic behavior of a real binary solution and an ideal solution 
are represented by the excess quantities, i.e.
    \begin{equation}
        G ^ {xs} _M = G _M - G ^{id} _M
    \end{equation}

or using equation (5)

    \begin{equation}
        G ^ {xs} _M = H _M - T S ^{xs} _M
    \end{equation}
with
    \begin{equation}
        S ^{xs} _M = S _M + R (c _A \ ln c_A + c _B \ ln c _B)
    \end{equation}\\



\section*{2. Observable indicators}


\subsection*{(a) Segregating liquid alloys}

    From the point of view of interatomic interactions, a binary alloy is either (i) an ordered
alloy, where unlike atoms are chosen as nearest neighbors over like atoms, or (ii) a
segregated alloy, where like atoms are chosen to pairs as nearest neighbors over unlike
atoms. Unfortunately, there is no direct way to distinguish the constituent atoms and
hence the identification of a nearest-neighbor pair of atoms is challenging. In this case
either the structural data or the experimental thermodynamic functions (such as activity, the heat of
mixing, excess Gibbs energy of mixing, excess heat capacity, etc) or other thermophysical
data (such as viscosity, diffusivity, density, surface tension, electrical resistivity, etc) are
supposed to extract information associated with interatomic interactions. Some of the
empirical criteria as well as microscopic parameters which are used to identify segregated
alloys are summarized below.\\
(a) Alloys displaying positive deviations from Raoult’s law.\\
(b) The heat of formation and the excess Gibbs energy of mixing are positive.\\
(c) The concentration fluctuation in the long-wavelength limit ($ S _{cc}(0) $) is greater than
    the ideal value.


    Table 1 provides a list of $G^{xs} _{M}$ , $H _M$ and $S^{xs} _{M}$ at the equiatomic composition 
of segregating liquid alloys which are arranged according to the type of their phase equilibria. Of these Bi–
Zn, Pb–Zn, Cu–Pb, Cd–Ga, Al–Bi, Al–In, Al–Pb, etc, exhibit liquid immiscibility. $G^{xs} _{M}$ and
$ H _M $ are comparatively large positive quantities. For these alloys only a few experimental
heat capacity data are available . The data, in general, show a decrease in Cp with increasing temperature. The energetic and structural effects in the
solution phase can be more directly seen by the excess heat capacity $\Delta  C _p $values:

    \begin{equation}
        \Delta  C _p = C_p(c) - c_A \ C_{p,A} - c_B \ C_{p,B} 
    \end{equation}
    \\
    \\
    \\
    \\
$\Delta  C _p $ are positive and indicate maximum values near to $ T _c$ and $c _c$ . \\


 
\begin{table}[t!]
\centering
\caption{Thermodynamic properties of liquid binary alloys at equiatomic composition displaying
segregation. The values are from Ref. [(i) Hultgren et al 1973, (iii) Yu 1994]}
 \begin{tabular}{|c c c c c c|} 
 \hline
 Alloys & Ref. & T (K) & $G ^ {xs} _ {M}$/RT & $H _M/RT$ & $S ^ {xs} _ {M}$/R \\ [0.5ex] 
 \hline\hline
 Al–Bi & (i) & 900 & 0.814 & 0.823 & 0.09 \\ 
 Al–In &     & 1150 & 0.54 & 0.49 & -0.05 \\
 Al–Pb &     & 1700 & 0.527 & 0.847 & 0.32 \\
 Bi–Ga & (ii) & 550 & 0.493 & 0.433 & -0.06 \\
 Bi–Zn &     & 880 & 0.36 & 0.60 & 0.24 \\ [1ex] 
 \hline
 \end{tabular}
\end{table}

    Since the pioneering work by Hume-Rothery and his coworkers (see, for example, Hume-
Rothery and Raynor 1954), a substantial effort has been assembled to identify the factors
impacting the alloying behavior of liquid metallic mixtures, such as the difference in atomic
sizes, valence differences, electronegativity differences, etc. For the sake of a brief perusal, 
we enroll the basic physical, thermochemical and structural properties of pure liquid metals
(near the melting points) in table 2 which are the components of the binary mixture of
table 1. These properties are also useful for further discussions. At this step, it is not
possible to single out any individual elemental properties which might be held reliable
for demixing of liquid alloys. Yet, the practical analyses recommend
that quantities such as atomic size, the heat of vaporization, and electronegativity together hold
the key to the knowledge  of the segregation or order in a liquid alloy.


\subsection*{(b) Thermodynamic properties }


    Some of the thermodynamic properties of equiatomic segregating liquid alloys are tabulated
in table 1. Here we intend to discuss briefly the salient features of the various experimental
techniques and the specific results that exist as a function of concentration and temperature.
The experimental methods used to obtain reliable thermodynamic data at constant pressure
as a function of composition and temperature are briefly summarized.The entropy of formation $S _M$ 
of an alloy can only be defined directly from $C _p$(c, T ) data 
    \begin{equation}
        S _M(c,T) = \int_{0}^{T}  \,\frac{\Delta {C _p(c,T)}}{T} dT 
    \end{equation}                    
    \\
    \\


    \begin{figure}
        \begin{adjustwidth}{-1cm}{}
        \centering
        \caption*{Some physical, chemical and structural properties of liquid metals (near melting
         temperature) are associated with the formation of segregating type metallic
        mixtures. m, atomic weight (1u = $1.66 \times 10^{-27} \ kg $ ); $T _m$ , melting point;
        $ \Omega $, volume; $ \Delta H _m $ , enthalpy of melting; $ \Delta H _v $ , enthalpy of 
        evaporation; $ \Delta S _m $ , entropy of
        melting; $ \Delta S _v $ , entropy of vaporization; $r _1$ nearest-neighbour distance; Z, 
        first shell coordination; $\Gamma $, surface tension; x, Pauling electronegativity value.
        (i) After Iida and Guthrie (1988), (ii) after Waseda (1980); (iii) after Pauling (1960) }
         \begin{tabular}{|l | l | l | l | l | l | l | l | l | l | l | l|} 
         \hline
         \multirow{4}{*}{\rotatebox{90}{Metals}}  &  \multirow{4}{*}{\rotatebox{90}{$m ^{(i)} \ (u)$}}  &  \multirow{4}{*}{\rotatebox{90}{$ T _m ^{(i)} \ (K)$}}  &  \multirow{4}{*}{\rotatebox{90}{$ \Omega ^{(i)} \ (10^{-6}m^3)gmol^{-1}$}}  & 
         \multirow{4}{*}{\rotatebox{90}{$\Delta H _m ^{(i)} \ (kJmol^{-1})  $}}  &  \multirow{4}{*}{\rotatebox{90}{$\Delta H _v ^{(i)} \ (kJmol^{-1}) $}}  &  \multirow{4}{*}{\rotatebox{90}{$\Delta S _m ^{(i)}   $}} 
         &  \multirow{4}{*}{\rotatebox{90}{$\Delta S _v ^{(i)}   $}}  &  \multirow{4}{*}{\rotatebox{90}{$r ^ {(ii)} _{1} $}}  &  \multirow{4}{*}{\rotatebox{90}{$ Z ^ {(ii)}  $}}  &
         \multirow{4}{*}{\rotatebox{90}{$ \Gamma ^ {(i)} $}}  &  \multirow{4}{*}{\rotatebox{90}{$ (x) ^ {(iii)} $}} \\ [0.5ex]
            & & & & & & & & & & & \\
            & & & & & & & & & & & \\
            & & & & & & & & & & & \\
            & & & & & & & & & & & \\
            & & & & & & & & & & & \\
            & & & & & & & & & & & \\
            
            
         \hline\hline
         Al & 26.98154 & 933.35           & 11.6  & 10.46 & 291 & 11.2 & 104  & 2.82 & 11.5 & 914 & 1.5 \\ 
         Bi & 208.9804 & 544.1$\pm 0.05$  & 20.80 & 10.88 & 179 & 20.0 & 97.4 & 3.38 & 8.8  & 378 & 1.9 \\
         Cd & 112.41   & 594.05           & 14.00 & 6.40  & 100 & 10.8 & 96.2 & 3.11 & 10.3 & 570 & 1.7 \\
         Ga & 69.72    & 302.93$\pm 0.005$& 11.40 & 5.59  & 270 & 18.4 & 100  & 2.82 & 10.4 & 718 & 1.6 \\
         In & 114.82   & 429.55           & 16.3  & 3.26  & 232 & 7.58 & 98.9 & 3.23 & 11.6 & 556 & 1.7 \\
         Pb & 207.2    & 600.55           & 19.42 & 4.81  & 178 & 8.02 & 88.0 & 3.33 & 10.9 & 458 & 1.8 \\
         Zn & 65.38    & 692.62           & 9.94  & 7.28  & 114 & 10.5 & 96.6 & 2.68 & 10.5 & 782 & 1.6 \\ [1ex] 
         \hline
         \end{tabular}
        \end{adjustwidth}
    \end{figure}


To obtain $S _M$ according to equations (28) and (29) the $C _p$ values of the components and the
alloy have to be known down to 0 K as well as the entropy of transformations that take
place below T . There are several problems. At first, the differences between the Cp valuesof the mechanical mixture and the alloy are small and one has to know the $ C _p$ values to 
high accuracy to get reliable results for $S _M$ . Equation (29) uses likewise to ordered
crystalline substances in the solid-state only. Alloys are occasionally disordered at room
temperature and remain so down to 0 K. $S _M$ values of a liquid alloy cannot be acquired from
equation (29) because the $C _p $  values of the undercooled liquid state for the components and
the alloys have to be verified specifically. Since for multicomponent systems, the reference 
state is the mechanical mixture of the pure components in the same state as the solution,
the entropy of formation of solid and liquid alloys are, hence, typically confined from
experimentally obtained $ G _M$(c, T) and $ H _M $(c, T) values according to equation (5).


\subsection*{(c) Calometric measurements}


The enthalpy of formation $ H _M $, their partial values  $ \overline{H} _{M,i} $
and the heat capacity of liquid alloys can be directly specified by calorimetric methods.
An isoperibolic type of calorimeter which operates at constant T is particularly appropriate for
calculating $H _M $ and $ \overline{H} _{M,i} $ of a liquid alloy as a function of composition 
at constant T directly.The $ H _M $ values acquired for liquid In–Cd alloys at 628 K are
shown in figure 2 as an example.
\begin{adjustbox}{center,caption={Enthalpy of mixing of liquid Cd–In alloys.},label={somelabel},nofloat=figure,vspace=\bigskipamount}
    %\includegraphics[width=0.7\textwidth]{fig-3.jpg}
\end{adjustbox}

If the modification in concentration $ \Delta c _A $ is small for each successive step (i.e. $ <  1 \ at.\% $),
$ dH _M (c)/dc $ can be confined in a suitable approximation by


    \begin{equation}
       \frac{ dH _M (c)}{dc _A} \left(c _A + \frac{\Delta c _A}{2} \right) = \frac{H _M 
       (c _A + \Delta c _A )-H _M (c _A)}{\Delta c _A } 
    \end{equation} 

The partial values of a multi-component system are obtained by

    \begin{equation}
        \overline{H} _{M,i} = H _M + \sum_{j = 2}^{r} (\delta _{ij} - c _j)\frac{\partial H _M{(c)}}{\partial c _j}   
    \end{equation} 

with $ \delta  _{ij} =0 $ for $i \neq  j$ and $ \delta  _{ij} =1 $ for $i = j$ . r is the 
number of components. Figure 3 shows the experimentally determined slope $ d H _M / dc _{C _d} $ 
of liquid In–Cd alloys at 628 K. These results undoubtedly show that small deviations from a standard solution behaviour $ (H _M (c) = Ac _Ac _B) $
exist.

The heat capacity of liquid alloys can be specified directly by adiabatic calorimetry.
Adiabatic calorimetry applies to calculate the heat input $ \Delta Q $ to a sample and the 
associated temperature increase $ \Delta T $.
Heat losses have to be minimized by proper surroundings to approximate an adiabatic
chamber for the sample. The specific heat over the temperature increase is given by 


    \begin{equation}
        C _p = \frac{\Delta Q}{m\Delta T} 
    \end{equation}   
where m is the mass of the sample.\\

\begin{adjustbox}{center,caption={Calorimetrically determined slope of the enthalpy of mixing of liquid Cd–In alloy at
    628 K (after Predel and Oehme 1977). },label={somelabel},nofloat=figure,vspace=\bigskipamount}
    %\includegraphics[width=0.7\textwidth]{fig 4.jpg}
\end{adjustbox}


\section*{3. Optimization of thermodynamic data}  

Thermodynamic calculations of phase equilibria are widely used to check the consistency
of data got from different experimental measurements (phase diagram data, results of
calorimetry ). Model descriptions using statistical
thermodynamics or polynomial expressions are used to represent the thermodynamic
properties of all phases applied.The adjustable coefficients are determined by a weighted
least-squares method (e.g. Lukas and Fries 1992). The essential feature of this procedure
is to obtain a uniform set of model parameters in an analytical representation. This helps
one to gather into temperature and concentration regions where the direct experimental
determination is difficult. It also allows one to estimate safely the thermodynamic data of
metastable phases. Finally, the thermodynamic description of multicomponent systems can
be gathered from those already calculated for their subsystems. The strategy of such a
critical assessment will be demonstrated for the demixing Al–In, Al–Pb, Cd–Ga, and Bi–Zn
systems. \\


\subsection*{(a) The Al–In System }

The $ H _M $ values near the equiatomic composition received by Predel and 
Sandig (1969a) are about 50\% more enormous than the values determined by 
Girard (1985) and Sommer et al (1993).The alloy samples, each of about 0.5 
g, were included in a closed graphite crucible which was encapsulated under 
argon in a quartz glass ampule. The quartz glass ampules were mechanically 
vibrated at about 1200 K to ensure a homogeneous liquid alloy before the DTA 
experiment on cooling was formed to obtain the binodal. The critical temperature 
amounted to 1112 K. \\
    In the background of this information, the optimization 
is performed. The major task is to characterize the thermodynamic properties 
as a power-series law whose coefficients
(say, A, B, C, D, . . .) are determined by the least-squares method. The 
heat capacity can be expressed as

    \begin{equation}
        C _p = -C - 2DT -2ET^{-2} - ....
    \end{equation}
The enthalpy and energy is given by\\

    \begin{equation}
        H = H(T _0) + \int_{0}^{T}  \,C _pdT
    \end{equation}
or\\
     \  $ H = A - CT -DT^{2} + 2ET^{-1} - .... $ \\
and
    \begin{equation}
        S = S(T_0) + \int_{0}^{T}  \,\frac{C _p}{T} dT 
    \end{equation}

or\\
\   $ S = -B -C(1 + lnT) - 2DT +ET^{-2} - ... $\\

Using equation (5), the T dependence of the Gibbs energy may be written as\\

    \begin{equation}
        G = A + BT + CT ln T +DT^2 + ET^{-1} + ... 
    \end{equation}

The composition dependency of the excess Gibbs energy of mixing 
is given by a polynomial such as the Redlich–Kister polynomial 
(Redlich and Kister 1948):

    \begin{equation}
        G^{xs}_M{(c, T)} = c _Ac _B\sum_{l = 0}^{m} K _l{(T)}(c_A - c _B)^l   
    \end{equation}
with $ K_l (T ) = A _l + B_l T + C_l T ln T + D_l T^{2} + ...$ The coefficients Kl have the same
temperature dependence as G in equation (37). The partial
 quantities are given by \\

    \begin{equation}
        \overline{G}^{xs}_A{(c, T)} = c _B^{2}\sum_{l = 0}^{m} K _l{(T)}[(1 + 2l)c_A - c _B](c_A - c _B)^{l-1}  
    \end{equation}
    \begin{equation}
        \overline{G}^{xs}_B{(c, T)} = c _A^{2}\sum_{l = 0}
        ^{m} K _l{(T)}[c_A - (1 + 2l)c _B](c_A - c _B)^{l-1}  
    \end{equation}

The pure solid elements Al and In in their stable form at 298.15 K 
and 1bar were chosen as the reference state of the system. 
The Gibbs energies of the elements as functions of the \\

\begin{adjustbox}{center,caption={calculated phase diagram 
    (continuous curve) using the coefficient set given in 
    table 2.
    },label={somelabel},nofloat=figure,vspace=\bigskipamount}
    %\includegraphics[width=0.7\textwidth]{fig 7.jpg}
\end{adjustbox}
\begin{table}[t!]
    \centering
    \caption{Optimized coefficient set of $G ^ {xs} _ {M}$ 
    (equation (37)) for liquid Al–In alloys}
     \begin{tabular}{|c c c|} 
     \hline
     l & $A_l \ (J mol^{-1})$ & $B_l \ (J mol^{-1}K^{-1})$ \\ [0.5ex] 
     \hline\hline
     0 & 18641.14 & 1.74886  \\ 
     1 & 558.36 & 1.14350  \\
     2 & 10692.88 & 7.47862  \\
     3 & 1346.56 & 0 \\ [1ex] 
     \hline
     \end{tabular}
    \end{table}
\begin{adjustbox}{center,caption={Calculated enthalpy of mixing 
    (continuous curve) using the coefficient set given in
    table 2.},label={somelabel},nofloat=figure,vspace=\bigskipamount}
    %\includegraphics[width=0.7\textwidth]{fig 8.jpg}
\end{adjustbox}

temperature were compiled by Dinsdale (1991) and no solid 
solubility was considered. The excess Gibbs energy of the 
liquid alloy is represented by equation (37). The optimized 
coefficients are given in table (3) and the phase diagram in 
figure (4).\\



\subsection*{(b) The Al–Pb system.}

The experimental results on Al–Pb imply a vast liquid
miscibility gap due to the strong segregation tendency of the components. The solubility of
lead in solid aluminum is less than 0.025 at.\% Pb at the monotectic temperature of around
932 K. The solubility of Al in Pb is basically negligible. A lot of phase diagram data are
available at temperatures below 1200 K in narrow terminal sides below 3 at.\% Pb and above 90 
at.\% Pb. These data are in acceptable agreement as the evaluation of McAlister (1984) has 
shown. McAlister has evaluated a critical temperature of 1839 K at 44.8 at.\% Pb. This $T _c$
value is considerably higher than the value specified by Predel and Sandig (1969b) operating
DTA. Yu et al (1996) have redetermined the binodal using a new isopiestic method (Wang et al 
1993). There are only a few thermodynamic data available, due to the experimental
difficulties associated with the small solubility of liquid aluminum and lead, and the high
vapor pressure of lead at high temperatures.\\ 
\begin{table}[t!]
    \centering
    \caption{Optimized coefficient set of $G ^ {xs} _ {M}$ 
    (equation (37)) for liquid Al–Pb alloys}
     \begin{tabular}{|c | c | c|} 
     \hline
     l & $A_l \ (J mol^{-1})$ & $B_l \ (J mol^{-1}K^{-1})$ \\ [0.5ex] 
     \hline\hline
     0 & 47993.6 & -10.71995  \\ 
     1 & 14407.33 & -6.65287  \\
     2 & 4742.36 & -0.72034  \\ [1ex] 
     \hline
     \end{tabular}
    \end{table}

The phase equilibria are calculated by choosing the pure elements in their stable state at 
298.15 K and 1 bar as the reference state of the system. Their Gibbs energies are given by 
Dinsdale (1991). The excess Gibbs energy of the liquid alloy is represented by equation (37). 
The calculation carried out by Yu et al (1996) takes into account the elemental thermodynamic 
data due to Dinsdale (1991), the phase diagram data at
temperatures above 1500 K that are obtained with the isopiestic method, and the data of Predel 
and Sandig (1969b) on the Pb-rich side below 1600 K. The resulting optimized set of parameters 
are given in table (4). The entire phase diagram is given in figure (6). 

\begin{adjustbox}{center,caption={Calculated phase diagram (continuous curve) using the 
    coefficient set given in table 3.},label={somelabel},nofloat=figure,vspace=\bigskipamount}
    %\includegraphics[width=0.7\textwidth]{fig 9.jpg}
\end{adjustbox}


\subsection*{(c) The Cd–Ga system.}

The Cd–Ga system shows a flat liquid miscibility gap with $T _c$ = 568 K and $c^{c}_{Ga}$ = 39.9 at.\% (see figure 12). 
The pure solid elements in their stable state at 298.15 K and 1 bar were chosen as
the reference state of the system and no solid solubility of the components was considered. The
Gibbs energies of the pure elements were taken from Dinsdale (1991). The excess Gibbs
energy of the liquid alloy was expressed with equation (37) and the resulting coefficient
set of the optimization is given in table 4. 
\begin{table}[t!]
    \centering
    \caption{Optimized coefficient set of $G ^ {xs} _ {M}$ 
    (equation (37)) for liquid Cd–Ga alloys}
     \begin{tabular}{|c | c | c | c | c |} 
     \hline
     l & $A_l \ (J mol^{-1})$ & $B_l \ (J mol^{-1}K^{-1})$ & $C_l \ (J mol^{-1}K^{-1})$ & $D_l \ (J mol^{-1}K^{-2})$\\ [0.5ex] 
     \hline\hline
     0 & -18447.76 & 483.09573 & -71.723197 & 0.041784  \\ 
     1 & -3189.49 & 38.38390 & -5.153091 & 0 \\
     2 & 3054.07 & -2.49129 & 0 & 0  \\ [1ex] 
     \hline
     \end{tabular}
    \end{table}

The calculated phase equilibria are shown in
figure 7. The experimentally determined temperature dependence of HM (see figure 8)
and the cadmium activity data of the liquid alloy (see figure 9) are consistent with the calculation.

\begin{adjustbox}{center,caption={Calculated phase diagram (continuous curve) using the coefficient set given in
    table 4.},label={somelabel},nofloat=figure,vspace=\bigskipamount}
    %\includegraphics[width=0.7\textwidth]{fig 12.jpg}
\end{adjustbox}
\begin{adjustbox}{center,caption={Calculated (continuous curve) enthalpy of mixing at different temperatures (1, 609 K;
    2, 656 K; 3,695 K) using the coefficient set given in table 4.},label={somelabel},nofloat=figure,vspace=\bigskipamount}
    %\includegraphics[width=0.7\textwidth]{fig 13.jpg}
\end{adjustbox}
\begin{adjustbox}{center,caption={Calculated (continuous curve) activities at 742 K using the coefficient set given in
    table 4.},label={somelabel},nofloat=figure,vspace=\bigskipamount}
    %\includegraphics[width=0.7\textwidth]{fig 14.jpg}
\end{adjustbox}

\subsection*{(d) The Bi–Zn system.}

The liquid zinc-rich Bi–Zn alloys exhibit a comprehensive miscibility
gap with $T _c$ = 863.8 K,  $c^{c}_{Zn}$  = 87 at.\% (see figure 10). At the monotectic temperature
of 688.5 K hcp zinc and two liquid alloys with the composition 38.6 and 99.1 at.\% Zn, 
respectively, are in equilibrium. These results are obtained from an optimization calculation
by the least-squares method (Lukas et al 1977). Preliminary results are given by Wang et al
(1993). No solid solubility of the components was assumed. The thermodynamic data for the
elements were assumed from Dinsdale (1991). The resulting set of coefficients describing $G^{xs}_M$
(equation (37)) of the liquid alloys is given in table 5.
\begin{table}[t!]
    \centering
    \caption{Optimized coefficient set of $G ^ {xs} _ {M}$ 
    (equation (37)) for liquid Cd–Ga alloys}
     \begin{tabular}{|c | c | c | c | c |} 
     \hline
     l & $A_l \ (J mol^{-1})$ & $B_l \ (J mol^{-1}K^{-1})$ \\ [0.5ex] 
     \hline\hline
     0 & 17633.89 & -7.91451  \\ 
     1 & -6607.32 & 1.34247 \\
     2 & 3873.47 & -1.08723 \\
     3 & 0 & 0\\
     4 & 7975.68 & -9.81903 \\
     5 & -4553.59 & 0 \\ [1ex] 
     \hline
     \end{tabular}
    \end{table}
A comparison between calculated $H _M$ and activity values and the experimental data are shown 
in figures 11 and 12.
\begin{adjustbox}{center,caption={Calculated (continuous curve) phase diagram using the coefficient set given in
    table 5. },label={somelabel},nofloat=figure,vspace=\bigskipamount}
    %\includegraphics[width=0.7\textwidth]{fig 17.jpg}
\end{adjustbox}
\begin{adjustbox}{center,caption={Calculated (continuous curve) enthalpy of mixing of liquid 
    Bi–Zn alloys using the
    coefficient set given in table 5. },label={somelabel},nofloat=figure,vspace=\bigskipamount}
    %\includegraphics[width=0.7\textwidth]{fig 18.jpg}
\end{adjustbox}
\begin{adjustbox}{center,caption={Calculated (continuous curve) activity of liquid Bi–Zn alloy at 873 K using the
    coefficient set given in table 5.  },label={somelabel},nofloat=figure,vspace=\bigskipamount}
    %\includegraphics[width=0.7\textwidth]{fig 19.jpg}
\end{adjustbox}

\section*{4. Electronic theory of mixing }

    The electronic theory in a major way provides a platform where the energies and structure of
a liquid metallic system can be merged (Harrison 1966, Heine 1970, Stroud and Ashcroft
1972) to remove absorbing physical properties of the system. 
There is no exact difference between nearly-free-electron (NFE) and non-NFE alloys. Yet, the 
physical properties of the two classes of alloys that are dependent on the number of valence 
electrons (electrical resistivity, thermoelectric power, Hall coefficient, magnetic 
susceptibility, knight shift) are usually extremely distinguishable. Also, the valence and 
the electronegativity differences are smaller in NFE and larger for the non-NFE systems.
The work on phase-separating liquid alloys based on electronic theory is insufficient,
although safe to apply to simple segregating liquid alloys without incurring noticeable error. 
The value of the theory stands from the fact that it is free from any assumptions
and, as such, there should not be an adaptable parameter. One can, at least, gather some
first-hand information on the basic interatomic forces ($\varphi  _{ij}$ ) which are directly related to the
energy parameter ($\omega $) occurring in previous sections. In addition, it helps to identify 
the volume and the structure-dependent contributions to the energy of the process of mixing.\\

\subsection*{(a) Pseudopotential perturbation scheme}

The NFE binary alloys can be considered to consist of a system of ions and valence electrons.
The fundamental Hamiltonian which explains the system is of the form\\

    \begin{equation}
        H = H_e + H_i + H_{ei}
    \end{equation}
with
    \begin{equation}
        H_e = T _{elec} + \frac{1}{2}\sum_{i = j}^{} \frac{e^2}{|\overline{r_i} - \overline{r_j}|} - H^{\prime }  
    \end{equation}
    \begin{equation}
        H_i = T_{ions} +\frac{1}{2}\sum_{i = j}^{} \frac{z_{i}z_{j}e^2}{|{R_i} - {R_j}|} - H^{\prime }  
    \end{equation}
    \begin{equation}
        H_{ei} = \sum_{i , j}^{} {V(\overline{r_i} - \overline{R_j})} + 2H^{\prime }  
    \end{equation}

(e, electron; i, ion; and ei, electron-ion). T stands for kinetic energy,$\overline{r}$and $\overline{}$are 
electronic and atomic positions and z is the valency. It is assumed that electrons and ions 
always interact among themselves Coulombically. The electron-ion interaction V will be accepted 
as a pseudopotential. Within an acceptable uncertainty, the ions are likely to drive much more 
slowly than the electrons, and, hence, the various contributions thus deriving from (40) can 
be treated individually. $H^{\prime}$  is the self-energy of a uniform charge distribution that is 
inserted to maintain the net potential energy expressions finitely. Referring to Hamiltonian 
(40), the internal energy, E, can be described in atomic units (for a detailed description
see Ashcroft and Stroud (1978)),

    \begin{equation}
       E = E_e + E_i + (E^{I} _{ei} + E^{II} _{ei})
    \end{equation}
with

\begin{align}
    &  E_e =  \overline{z}\left[\frac{3}{10} K^{2} _F - \frac{3}{4\pi} K_F - 0.0474 - 0.0155 
        ln {K_F} - \frac{1}{2}\left(\frac{\pi k_B}{K_F} \right)^{2} T^2 \right] \\
    &  E_i =  \frac{3}{2}k_{B}T + \frac{1}{\pi}\sum_{i , j}^{A , B} z_{i}z_{j}\left(c_{i}c_{j}\right)^{\frac{1}{2}}
        \int_{0}^{\infty}  \,\left(S_{ij}(q) - \delta _{ij}\right) dq \\   
    & E^{II}_{ei} = \lim_{q \to 0} \overline{z}\varrho 
       \left[\sum_{i}^{} c_{i}V_{i}(q) + \frac{4\pi \overline{z}}{q^2} \right]  \\ 
    &  E^{II}_{ei} = \frac{1}{16{\pi}^3}\int_{0}^{\infty}  \,q^{4}dq \sum_{ij}^{} V_{i}(q)V_{j}(q) 
        \left(c_{i}c_{j}\right)^{\frac{1}{2}}S_{ij}(q)\left(\frac{1}{\epsilon^{ \ast}(q)} - 1 \right) 
    \end{align}\\

where $K_F = \left( 3\pi ^{2}\overline{z}\varrho\right)^{\frac{1}{3}}$ , $\overline{z}\varrho = z_{A}\varrho_{A} + z_{B}\varrho_{B}$ and $\overline{z} = c_{A}z_{A} + c_{B}z_{B}$
; $z_A$ and $z_B$ are valencies, $\varrho _A$ and $\varrho _B$ are the number densities of the ion 
species and $\varrho = \varrho _A + \varrho _B $ . $E^{I}_{ei}$ and $E^{II}_{ei}$ are due to
the electron-ion interaction specified through first and second-order pseudopotential 
perturbation
theory, respectively. $V(q)$ is the Fourier transform of the bare ion pseudopotential and is
named the form factor. $\epsilon ^{\ast}(q)$ is the altered Hartree dielectric function which brings into 
account the interaction of the conduction electrons.\\

    $S_{ij}$ are the partial structure factors that take care of the arrangement of ions in the
system. For a hard-sphere system,  $S_{ij}$ can readily be calculated following the work by
Ashcroft and Langreth (1967). The necessary inputs are the diameters of the hard spheres ($\sigma $)
whose preference has always been a matter of interest (see, for example, Faber 1972, Shimoji
1977). However, to make the present scheme internally consistent, we suggest that $\sigma$ should be 
specified in the variational thermodynamic purpose equipping minimum free
energy for the system, i.e.
    \begin{equation}
       \left(\frac{\partial F}{\partial \sigma }\right) _{\Omega ,T} = 0
    \end{equation}
with\\
    \begin{equation}
        F = E - TS_{hs}
    \end{equation}
where $S_{hs}$ is the entropy of the hard sphere mixture which consists of \\
    \begin{equation}
        S_{hs} = S_{id} + S_{gas} + S_{\eta} + S_{\sigma}
    \end{equation}
where $S_{id}$ is the ideal entropy of mixing, $S_{gas}$ is the ideal gas entropy, $S_{\eta}$ 
is the contribution
which depends only on packing density and $S_{\sigma}$ denotes the entropy contribution due to
mismatch of the hard sphere diameters $\sigma_1$ and $\sigma_2$ .\\
    The operating expressions for these quantities may be described as (Mansoori et al 1971,
Umar et al 1974): 
   

\begin{align}
    &  S_{id} = -k_B \sum_{i = 1}^{2} c_{i} lnc_{i} \\
    &  S_{gas} = \frac{5}{2}k_{B} + k_{B}ln\left[\frac{1}{\varrho}\left(\frac{m^{c_{A}}_{A}m^{c_{B}}_{B}k_{B}T}{{2\pi \hbar ^{2}}}\right)^{\frac{3}{2}} \right] \\
    &  S_{\eta} = k_B lnb + 1.5k_{B}(1 - b^2)\\
    & S_{\sigma} = \frac{\pi c_{A}c_{B}\varrho(\sigma_{A} - \sigma_{B})^{2}b^{2}}{24}[12(\sigma_{A} + \sigma_{B}) -
     \pi \varrho (c_{A}\sigma^{4}_{A} + c_{B}\sigma^{4}_{B})]     
    \end{align}\\

with $b = (1 - \eta)^{-1}$ . The first two terms in equation (51) are structure-independent terms and 
depend only on concentration c, atomic mass m, and atomic volume $\Omega $. The last two terms are 
structure-dependent contributions due to the existence of the packing
fraction, $\eta (= (\pi \varrho/6)(c_A\sigma^3_A + c_B\sigma^3_{B})) $ and the hard-sphere diameter $\sigma _{i}$ .\\
    The present method helps to specify the energy contributions from the constituent
species such as electrons and ions. Also, it helps us to separate the energies of the formation
of a binary alloy due to just volume-dependent and structure-dependent contributions,
respectively.\\
The Hamiltonian as expressed in equation (40) allows (see, for example, Ashcroft and Stroud 1978, Young
1992) one to estimate the effective interatomic potentials as,
    \begin{equation}
        \phi _{ij}(r) = \frac{z_{i}z_{j}e^2}{r} + \frac{1}{(2\pi)^3}\int_{}^{}  \,\frac{q^2}{4\pi e^2}V_{i}(q)V_{j}(q)
        \left[\frac{1}{\epsilon ^{\ast}(q) } - 1\right]\frac{\sin qr}{qr} 4\pi q^2 dq 
    \end{equation}
$\phi _{ij}(r)$ can readily be coupled to the radial distribution function, $g_{ij}(r)$.\\

\section*{5. Hard-sphere like theory for segregation}

\subsection*{(a) Hard sphere mixture under the Percus–Yevick approximation}

The stability of binary alloys by treating them as a mixture of hard spheres, i.e.

\begin{equation}
    \phi (r) = \infty    %\tag*{for r < \sigma      \phi (r) = 0    r > \sigma}
\end{equation}

Although the potential (equation (57)) is devoid of an attractive interaction, it equips a 
useful insight into which forms the effective short-range repulsive interaction governs the 
geometrical packing at metallic densities. The quantity of major significance is the direct
correlation function, $C_{ij}(r)$ , which is connected to $g_{ij}(r)$ and $\phi_{ij}(r)$ via the Percus–Yevick
(PY) equation (Percus and Yevick 1958):
    \begin{equation}
        C_{ij}(r) = g_{ij}(r)\left(1 - exp\frac{\phi_{ij}}{k_{B}T}\right)          \tag*{(i , j = A , B)}
    \end{equation}
The analytical solution of the PY equation for a mixture of hard spheres was obtained by 
Lebowitz (see Lebowitz 1964, Lebowitz and Rowlinson 1964). As regards the hard-core 
interactions in the mixture, these authors considered additive hard-sphere mixtures, i.e
    \begin{equation}
        \sigma_{AB} = \frac{\sigma_{AA} + \sigma_{BB} }{2}
    \end{equation}
The free energy per particle of the mixture can be written as
\begin{align}
    &  F_{hs} = c_{A}\mu_{A} + c_{B}\mu_{B} - P _{hs}\Omega \\
    &  G_{hs} = c_{A}\mu_{A} + c_{B}\mu_{B}
    \end{align}
Here the $\mu_i$ are the chemical potentials of the components and $P _{hs}$ is the pressure:

    \begin{equation}
        \frac{\mu_i}{k_{B}T} = ln \left[\Omega ^{-1}_{i}\left(\frac{2\pi \overline{h}^2}
        {m_{i}k_{B}T}\right)^(3/2)\right] - ln (1 - \eta) + \frac{3X\sigma_{ij}}{(1 - 
        \eta)} + \frac{3}{2}\left[\frac{3X^2}{(1 - \eta )^2} + \frac{2Y}{(1 - \eta)}\right] 
        \sigma^2_{ii} + \frac{\pi P _{hs}\sigma^3_{ii}}{6k_{B}T}
    \end{equation}


\end{document}