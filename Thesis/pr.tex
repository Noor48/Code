\documentclass[12pt, letterpaper]{article}
\usepackage{bbold}
\usepackage{indentfirst}
\usepackage{amsmath, amssymb}
\usepackage[T1]{fontenc}
\usepackage[utf8]{inputenc}
\usepackage{physics}
\usepackage{tensor}
\usepackage{braket}
\usepackage{graphics}
\usepackage{grffile}
\usepackage{float}
\usepackage{adjustbox}

\begin{document}
    Theoreticians have reported the importance of the mixing properties of binary liquid alloys from both the scientific and the technological points of view. 
    An accurate knowledge of the mixing properties and phase diagrams of the alloy system are essential to establish a good understanding between the experimental 
    results, theoretical approaches and empirical models for liquid alloys with miscibility gap.all liquid binary alloys can be grouped into two distinct classes that 
    either exhibit positive deviation
    (usually called segregating systems) or negative deviation (i.e. short-ranged ordered alloys) from Raoult’s law or the additive rule of mixing. If the deviations 
    are considerably large,they may lead either to phase separation or compound formation in the binary system.There are liquid alloys, however, which do not belong 
    exclusively to any of the above two classes. For example, the excess Gibbs energy of mixing ($G ^{xs} _{M} $) for Cd–Na and Ag–Ge is negative at certain compositions, 
    while positive at other compositions. In liquid alloys such as Au–Bi, Bi–Cd and Bi–Sb, the enthalpy of mixing (HM) is a positive quantity but $G^{xs}_{M}$ is negative. 
    Bi–Pb has positive HM and $G^{xs} _{M}$ in the solid phase as against the negative xs values of H M and G M in the liquid phase. Systems such as Au–Ni and Cr–Mo exhibit 
    immiscibility in the solid phase which is not visible in the corresponding liquid phase. Systems such as Ag–Te show inter metallic phases and large negative H M 
    values in the liquid phase together with a liquid miscibility gap.

    Systems such as Al–Bi, Al–In, Al–Pb, Bi–Ga, Bi–Zn, Cd–Ga, Ga–Pb, Ga–Hg, Pb–Zn, Pb–Si and Cu–Pb, etc, are characterized by liquid miscibility gaps and exhibit large 
    positive HM. Their properties in the liquid phase tend to change markedly as a function of composition (c), temperature (T ) and pressure (p). The long wavelength 
    limit (q → 0)of the composition–composition structure factors, S cc (0) diverges as the composition and temperature approach the critical values c → c c , and T → Tc .
    Scc (0), which can directly be obtained from thermodynamic functions (either obtained by taking the first composition derivative of the activity or through the 
    second derivative of the Gibbs function), is very useful for ascertaining the immiscibility and the degree of segregation in binary liquid alloys. Other 
    thermodynamic, structural and transport properties are also found to change anomalously in the vicinity of c c and Tc .
\end{document}