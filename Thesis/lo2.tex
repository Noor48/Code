\documentclass[12pt, letterpaper]{article}
\usepackage{bbold}
\usepackage{indentfirst}
\usepackage{amsmath, amssymb}
\usepackage[T1]{fontenc}
\usepackage[utf8]{inputenc}
\usepackage{physics}
\usepackage{tensor}
\usepackage{braket}
\newcommand*{\1}{\hspace{1pt}}
\author{Noor E Mustafa Ferdous}



\title {A Course on Quantum Field Theory}

\date{8th June, 2021}

\begin{document}
    \maketitle

    \section*{Time Ordered Product}

    Purpose of a picture is initial condition of a system 
    From scalar quantization 

    \begin{equation}
        [\ \tensor \phi{_i}, \, \pi{_j} \, ] = \, i\hbar \delta{_{ij}}
    \end{equation}
    
     $\phi (x)$ in 3D
    
    \begin{equation}
        \implies    [\ \phi(x), \, \pi(x ^{\prime}) \, ] = \, i \delta ^{3}(x-x ^{\prime})
    \end{equation}

    which is more fundamental apporach for quantization. But in 1958, R. Haggs gave us a theorem that "Interaction
    picture does not exists in QFT".

    In Dyson picture, the wave function will be
    \begin{equation}
        \ket{\psi (t)} _{I} = U(t,t_{0})\ket{\psi (t_{0})}
    \end{equation}
    
    The expression of U is 

    \begin{equation*}
        U(t,t_{0})  = 
            \begin{cases}
            e^{iH(t-t_{0})}, & \text {if H is time independent} \\
            Te^{-i\int_{t{_0}}^{t} H(t^{\prime}) \,dt^{\prime} }, & \text {if H is time dependent}
            \end{cases}
    \end{equation*}

    if H is time dependent, We need time ordered products
    
    \begin{equation}
        \begin{split}
            T[A(t)B(t^{\prime})] & =
            \begin{cases}
                A(t)B(t^{\prime}), & \text{if } t > t^{\prime} \\
                B(t^{\prime})A(t), & \text{if } t^{\prime} > t
            \end{cases} \\
            \implies T[A(t)B(t)^{\prime}] & = \theta (t-t^{\prime})A(t)B(t^{\prime}) + \theta B(t^{\prime})A(t)
        \end{split}
    \end{equation}

    For a real scalar field:
    \begin{equation*}
        \pi(t) = \dot{\phi(t)}
    \end{equation*}

    \begin{equation}
        \begin{split}
            \frac{d^2 }{d t^2}T[\phi (t) \phi (t^{\prime})] & = \frac{d^2 }{d t^2}[\theta (t-t^{\prime})\phi (t) \phi (t^{\prime}) + \theta (t^{\prime} -t) \phi(t^{\prime}) \phi (t)] \\
            & =  \frac{d}{d t}[\frac{d \theta}{d t}\phi (t) \phi (t^{\prime}) + \theta (t-t^{\prime})\frac{d \phi (t)}{dt}\phi (t^{\prime}) \\
            & \ \ \ \ + \frac{d\theta (t^{\prime} -t)}{dt}\phi (t) \phi (t^{\prime}) + \theta (t^{\prime} -t) \frac{d\phi(t)}{dt} \phi(t^{\prime})] \\
            & =  \frac{d}{d t}[\delta(t-t^{\prime})\phi (t) \phi (t^{\prime}) + \theta (t-t^{\prime})\dot{\phi} (t)\phi (t^{\prime}) \\
            & \ \ \ \ - \delta (t^{\prime} -t) \phi (t) \phi (t^{\prime})  +\theta (t^{\prime} -t) \dot{\phi}(t) \phi(t^{\prime})] \\
            & =  [\delta ^{\prime}(t-t^{\prime})\phi (t) \phi (t^{\prime}) + \delta (t-t^{\prime})\dot{\phi} (t)\phi (t^{\prime}) \\
            & \ \ \ \ + \delta (t^{\prime} -t) \dot{\phi} (t) \phi (t^{\prime}) + \theta (t^{\prime} -t) \ddot{\phi}(t) \phi(t^{\prime})] \\
            & \ \ \ \ + \delta ^{\prime} (t^{\prime} -t) \phi (t) \phi (t^{\prime}) - \delta (t^{\prime} -t) \dot{\phi}(t) \phi(t^{\prime})] \\
            & \ \ \ \ - \delta (t^{\prime} -t) \dot{\phi} (t) \phi (t^{\prime}) - \theta (t^{\prime} -t) \ddot{\phi}(t) \phi(t^{\prime})] \\
            & = \delta (t-t^{\prime})[\dot{\phi} (t), \phi (t)] \\
            & = i \delta ^{4} (t-t^{\prime})
        \end{split}
    \end{equation}

    Consider a real scalar field whose Lagrangian is
    \begin{equation}
        \mathcal{L}  = \frac{1}{2} (\partial \phi \partial \phi - m^{2} \phi ^{2})
    \end{equation}

    Equation of motion will be 
    \begin{equation}`
        (\Box + m^{2}) = 0
    \end{equation}

    The time ordered product 
    \begin{equation}
        \begin{split}
            \bra 0  T[\phi _{1}, \phi _{2}]  \ket 0 =  &  \ \theta (t_{1}-t_{2}) \bra  \phi(x_{1},t_{1}) \phi (x_{2}, t{_2})  \ket 0 \\
            & + \theta (t_{2}-t_{1}) \bra  \phi(x_{2},t_{2}) \phi (x_{1}, t{_1})  \ket 0
        \end{split}
    \end{equation}

    Assigment 1: Show that 
    \begin{equation}
        (\Box + m^{2}) \bra {0} T(\phi _{1}, \phi _{2}) \ket {0} = -i \delta^{4} (x_{1}-x_{2})
    \end{equation}

    defination of Green's function from a linear operator $\hat L$
    \begin{equation}
        \begin{split}
            \hat {L} \psi & = 0 \\
            \implies \hat L G_{ij} & = \delta_{ij} \\
            \implies \hat L G(x-y) & = iG(x-y) \\
            \therefore \bra {0} T(\phi_{1}, \phi_{2}) \ket {0} & = iG(x-y)  
        \end{split}  
    \end{equation}

    Assignment 2:

    \begin{equation}
        (\Box + m^{2})\bra {0} T[\phi_{1}, \phi _{2}, \phi _{3}] \ket {0} = 
    \end{equation}

    Find the RHS

    \begin{equation}
    \begin{split}
        T[ABC] = &  \ \theta(t_{2} - t_{3})\theta (t_{1} - t_{2})A(t_{1})B(t_{2})C(t_{3}) + \theta(t_{3} - t_{1})\theta (t_{2} - t_{3})B(t_{2})C(t_{3})A(t_{1}) \\
        & \ \  + \theta(t_{1} - t_{3})\theta (t_{1} - t_{2})C(t_{3})A(t_{1})B(t_{2}) - \theta(t_{3} - t_{2})\theta (t_{1} - t_{3})A(t_{1})C(t_{3})B(t_{2}) \\
        & \ \  - \theta(t_{2} - t_{1})\theta (t_{3} - t_{2})C(t_{3})B(t_{2})A(t_{1}) - \theta(t_{1} - t_{3})\theta (t_{2} - t_{1})B(t_{2})A(t_{1})C(t_{3}) \\ 
    \end{split}
    \end{equation}

    From Electrodynamics, there are two types of Green's funtion,
    \begin{equation}
        \begin{cases}
            G_{ret} = 0 & \text {if} x_{0} < y_{0} \\
            G_{ret} = 0 & \text {if} x_{0} > y_{0}
        \end{cases}
    \end{equation}
    
    $y_{0}$ is the location of cause, and $x_{0}$ is the obervers' position

    Lightcone picture:

    For free system
    \begin{equation}
        (\Box + m^{2}) \phi = 0
    \end{equation}

    For non free system
    \begin{equation}
        (\Box + m^{2}) \phi = J
    \end{equation}

    Where the Lagrangian will be
    \begin{equation}
        \mathcal{L}  = \mathcal{L} _{0} +J\phi
    \end{equation}
    
    Where
    \begin{equation}
        \mathcal{L} _{int}  = -\mathcal{H} _{int} 
    \end{equation}

    \begin{equation}
        \ket {out} = S \ket{in}
    \end{equation}

    \begin{equation}
        S = \lim_{t \to \infty \ t_{0} \to \infty} U(t,t_{0})  
    \end{equation}
    
\end{document}