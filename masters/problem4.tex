\documentclass[12pt, letterpaper]{article}
\usepackage{amsmath}
\usepackage{tensor}
\usepackage{stix}
\usepackage{slashed}
\usepackage{hyperref}
\usepackage[compat=1.0.0]{tikz-feynman}
\usepackage{feynmp-auto}
\usepackage{float}
\usepackage{graphicx} % Required for inserting images
%\usepackage[paperheight=6in,
%   paperwidth=5in,
%   top=10mm,
%   bottom=20mm,
%   left=10mm,
%   right=10mm]{geometry}
\newcommand*{\1}{\hspace{1pt}}
\title{Solution to Problem Sheet 4}
\author{Noor E Mustafa Ferdous\\ email: \href{mailto:nooremf@gmail.com}{nooremf@gmail.com} }

\date{}

\begin{document}

\maketitle

\section*{Solve for problem no. 1}
(a)
\begin{figure}[h]
\[
    \feynmandiagram [horizontal=a to b] {
        i1 [particle=\(\psi\)] -- [fermion, momentum'=\(p_{1}\)] a -- [scalar, reversed momentum'=\(p_{2}\)] i2 [particle=\(\phi\)],
        a -- [fermion] b,
        f1 [particle=\(\psi\)] -- [anti fermion, reversed momentum'=\(p_{3}\)] b -- [scalar, momentum'=\(p_{4}\)] f2 [particle=\(\phi\)],
      };
    +
    \feynmandiagram [horizontal=a to b] {
      i1 [particle=\(\phi\)] -- [scalar, momentum'=\(p_{1}\)] a -- [anti fermion, reversed momentum'=\(p _{2}\)] i2 [particle=\(\psi\)],
      a -- [fermion] b,
      f1 [particle=\(\psi\)] -- [anti fermion, reversed momentum'=\(p _{3}\)] b -- [scalar, momentum'=\(p _{4}\)] f2 [particle=\(\phi\)],
    };
  \]
  \caption{\label{fig:fi} Feynman diagram for $\psi\psi \to \psi\phi$.}
\end{figure}

Scattering amplitude for this diagram is 
\begin{align*}
  i \mathcal{M} &= \int \frac{d ^{4}k}{(2\pi)^{4}} \delta^{4} (p_{1} - p_{2} - k) \delta^{4} (p_{3} - p_{4} - k) \frac{i (2\pi)^{8}}{k^{2}-M^{2} + i \epsilon} \\
  &= (ig)^{2} i (2\pi)^{4} \delta ^{4} (p_{3} + p _{4} - p_{1} - p_{2}) \frac{1}{(p_{1} + p_{2})^{2} - M^{2} + i \epsilon}
\end{align*}

%\begin{fmffile}{first-diagram}
%  \begin{fmfgraph}(120,80)
%    \fmfleft{i1,i2}
%    \fmfright{o1,o2}
%    \fmf{fermion}{i1,v1,o1}
%    \fmf{fermion}{i2,v2,o2}
%    \fmf{fermion}{v1,v2}
%  \end{fmfgraph}
% \end{fmffile}
(b)
\begin{figure}[H]
  \[
      \feynmandiagram [horizontal=a to b] {
          i1 [particle=\(\psi\)] -- [fermion, momentum'=\(p_{1}\)] a -- [scalar,  reversed momentum'=\(p_{2}\)] i2 [particle=\(\phi\)],
          a -- [fermion] b,
          f1 [particle=\(\psi^{\dagger}\)] -- [fermion,  reversed momentum'=\(p_{3}\)] b -- [scalar, momentum'=\(p_{4}\)] f2 [particle=\(\phi\)],
        };
      +
      \feynmandiagram [horizontal=a to b] {
        i1 [particle=\(\phi\)] -- [scalar, momentum'=\(p_{1}\)] a -- [anti fermion, reversed momentum'=\(p _{2}\)] i2 [particle=\(\psi\)],
        a -- [fermion] b,
        f1 [particle=\(\psi^{\dagger}\)] -- [ fermion, reversed momentum'=\(p _{3}\)] b -- [scalar, momentum'=\(p _{4}\)] f2 [particle=\(\phi\)],
    };
    \]
    \caption{\label{fig:fi} Feynman diagram for $\psi\psi^{\dagger} \to \phi\phi$.}
  \end{figure}

The scattering amplitude will be
\begin{align*}
  i\mathcal{M} &= (-ig)^{2}\int \frac{d^{4} k}{(2\pi)^{4}}\frac{i (2\pi)^{8}}{k^{2}-M^{2}+i\epsilon}(\delta^{4}(p_{1} + p_{2} - k) \delta^{4}(k - p_{3} - p_{4})) \\
  &= i(-ig)^{2} (2\pi)^{4} \delta^{4} (p_{1} + p_{2} - p_{3} - p_{4}) \frac{1}{(p_{1} + p_{2}^{2}-M^{2}+i\epsilon)}
\end{align*}

\section*{Solve for problem no. 3}
\begin{figure}[H]
  \[
          \feynmandiagram [vertical=f2 to f3] {
          f1 [particle=\(e^{-}\)] -- [fermion, momentum'=\(p_{2}\)] f2 -- [fermion,  reversed momentum'=\(q\)] f3 -- [fermion, momentum'=\(p_{3}\)] f4 [particle=\(e^{-}\)],
          f2 -- [photon, momentum'=\(p_{4}\)] p1,
          f3 -- [photon,reversed momentum'=\(p_{1}\)] p2,
          };
          \feynmandiagram [horizontal=f2 to f3] {
          f1 [particle=\(e^{-}\)] -- [fermion, momentum'=\(p_{1}\)] f2 -- [fermion, momentum'=\(q\)] f3 -- [fermion, momentum'=\(p_{4}\)] f4 [particle=\(e^{-}\)],
          f2 -- [photon,reversed momentum'=\(p_{2}\)] p1,
          f3 -- [photon,momentum'=\(p_{3}\)] p2,
          };
      \]
      \caption{\label{fig:fi} Feynman diagram for $e^{-}\gamma \to e^{-}\gamma$}
  \end{figure}
(a)
Scattering amplitude will be 
\begin{align*}
  i\mathcal{M}_{1} &= (2\pi)^{4} \int d^{4} q \big[\overline{u}(p_4)(-ie\gamma^{\mu}\epsilon_{\mu}(p_{2}))\frac{i(\slashed{q} +m)}{q^{2}-m^{2}+i\epsilon}\epsilon^{*}_{\nu}(p_3)u(p_{1})\delta^{4}(p_{1} - p_{3} - q)\delta^{4}(p_{2}+q-p_{4})\big] \\
  \mathcal{M}_{1} &= \frac{e^{2}}{(p_{1}-p_{3})^{2}-m^{2}+i\epsilon}\big[\overline{u}(p_{4})\gamma^{\mu}\epsilon_{\mu}(p_{2})\epsilon^{*}_{\mu}(p_{3})\gamma_{\mu}u(p_{1})\delta^{4}(p_{2}+p_{3}-p_{1}-p_{4}))\big]
\end{align*}
and 
\begin{align*}
  i\mathcal{M}_{2} &= (2\pi)^{4} \int d^{4} q \big[\overline{u}(p_4)(-ie\gamma^{\mu}\epsilon_{\mu}(p_{2}))\frac{i(\slashed{q} +m)}{q^{2}-m^{2}+i\epsilon}\epsilon^{*}_{\nu}(p_3)u(p_{1})\delta^{4}(p_{1} + p_{2} - q)\delta^{4}(p_{4}+p_{3}-q)\big] \\
  \mathcal{M}_{2} &= \frac{e^{2}}{(p_{1}-p_{3})^{2}-m^{2}+i\epsilon}\big[\overline{u}(p_{4})\gamma^{\mu}\epsilon_{\mu}(p_{2})\epsilon^{*}_{\mu}(p_{3})\gamma_{\mu}u(p_{1})\delta^{4}(p_{4}+p_{3}-p_{1}-p_{2})\big]
\end{align*}
\begin{figure}[H]
\[
  \feynmandiagram [vertical=a to b] {
      i1 [particle=\(e^{-}\)] -- [anti fermion] a -- [anti fermion] i2 [particle=\(e^{+}\)],
      a -- [photon, edge label=\(\gamma\),  reversed momentum'=\(q\)] b,
      f1 [particle=\(e^{+}\)] -- [anti fermion] b -- [anti fermion] f2 [particle=\(e ^{-}\)],
    };
    \feynmandiagram [horizontal=a to b] {
      i1 [particle=\(e^{-}\)] -- [fermion] a -- [fermion] i2 [particle=\(e^{-}\)],
      a -- [photon, edge label=\(\gamma\), momentum'=\(q\)] b,
      f1 [particle=\(e^{+}\)] -- [fermion] b -- [fermion] f2 [particle=\(e ^{+}\)],
    };
  \]
      \caption{\label{fig:fi} Feynman diagram for $e^{-}e^{+} \to e^{-}e^{+}$}
\end{figure}
(b)
Scattering amplitudes will be 
\begin{align*}
  i \mathcal{M}_{1} &= \int \overline{u}(p_{3})(-ie\gamma^{\mu})u(p_{1})\frac{i \eta_{\mu\nu}}{q^{2}+i\epsilon}\overline{v}(p_{2})(-ie\gamma_{nu}) v(p_{4}) (2\pi)^{8}\delta^{4}(p_{1}-p_{3}-q)\delta^{4}(p_{2}+q-p_{4}) \frac{d^{4}q}{(2\pi)^4} \\
  &= -\frac{ie^{2}}{(p_{1}-p_{3})^{2}+i\epsilon}\overline{u}(p_{3})\gamma^{\mu}u(p_{1})\overline{v}(p_{})\gamma_{\mu}v(p_{4})
\end{align*}
\begin{align*}
 -i\mathcal{M}_{1} &= \frac{ie^{2}}{(p_{1}-p_{3})^{2}+i\epsilon}[\overline{u}(p_{3})\gamma^{\mu}u(p_{1})]^{\dagger}[\overline{v}(p_{2})\gamma_{\mu}v(p_{4})]^{\dagger} \\
 &= \frac{ie^{2}}{(p_{1}-p_{3})^{2}+i\epsilon}\overline{u}(p_{1})\gamma^{\mu}u(p_{3})\overline{v}(p_{4})\gamma_{\mu}v(p_{2})
\end{align*}

\begin{align*}
  i \mathcal{M}_{2} &= \int \overline{u}(p_{3})(-ie\gamma^{\mu})\overline{v}(p_{4})\frac{i \eta_{\mu\nu}}{q^{2}+i\epsilon}\overline{v}(p_{2})(-ie\gamma_{nu}) v(p_{4}) (2\pi)^{8}\delta^{4}(p_{1}+p_{2}-q)\delta^{4}(q-p_{3}-p_{4}) \frac{d^{4}q}{(2\pi)^4} \\
  &= - \frac{ie^{2}}{(p_{1}+p_{2}+i\epsilon)}\overline{u}(p_{3})\gamma^{\mu}\overline{v}(p_{4})\overline{v}(p_{2})\gamma_{\mu}v(p_{4})
\end{align*}

\begin{align*}
  -i \mathcal{M}_{2}&= \frac{ie^{2}}{(p_{1}+p_{2}+i\epsilon)}\overline{v}(p_{3})\gamma^{\mu}\overline{u}(p_{4})\overline{u}(p_{2})\gamma_{\mu}v(p_{4})
\end{align*}

\begin{align*}
  |\mathcal{M}|^{2} &= \frac{e^{4}}{(p_{1}-p_{3})^{4}}Tr[\overline{u}(p_{3})u(p_{3}\gamma^{\mu}u_{p_{1}}\overline{u(p_{1})})]Tr[v(p_{2})\overline{v(p_{2})}\gamma^{\mu\prime}\overline{v(p_{4})}v(p_{4})] \\
  &= \frac{e^{4}}{(p_{1}-p_{3})^{4}} [Tr(\slashed{p_{3}}\gamma^{\mu\prime}\slashed{p_{1}}\gamma^{\mu})+m^2Tr(\gamma^{\mu\prime}\gamma^{\mu})][Tr(\slashed{p_{2}}\gamma_{\mu\prime}\slashed{p_{4}}\gamma_{\mu})-m^2Tr(\gamma_{\mu\prime}\gamma_{\mu})] \\
  &= \frac{e^{4}}{(p_{1}-p_{3})^{4}}[p^{\mu\prime}_{3}p_{1}-p_{3}p_{1}\eta^{\mu\mu\prime} + p^{\mu}_{3}p^{\mu\prime}_{1} + m^{2}\eta^{\mu\mu}]\times[p_{2\mu\prime}p_{4}-p_{2}p_{4}\eta_{\mu\mu\prime} + p_{2\mu}p_{4\mu\prime} - m^{2}\eta_{\mu\mu}]
\end{align*}


\end{document}