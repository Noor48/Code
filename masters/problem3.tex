\documentclass[12pt, letterpaper]{article}
\usepackage{amsmath}
\usepackage{tensor}
\usepackage{stix}
\usepackage{slashed}
\usepackage{hyperref}
\usepackage{graphicx} % Required for inserting images
%\usepackage[paperheight=6in,
%   paperwidth=5in,
%   top=10mm,
%   bottom=20mm,
%   left=10mm,
%   right=10mm]{geometry}
\newcommand*{\1}{\hspace{1pt}}
\title{Solution to Problem Sheet 2}
\author{Noor E Mustafa Ferdous\\ email: \href{mailto:nooremf@gmail.com}{nooremf@gmail.com} }

\date{}

\begin{document}

\maketitle

\section*{Solve for problem no. 1}

Given
\begin{align*}
    \mathcal{L} &= -\frac{1}{4}F_{\mu \nu} F^{\mu \nu} - j^{\mu} A^{\mu} \\
    &= -\frac{1}{4}(\partial_{\mu}A_{\nu} - \partial_{\nu}A_{\mu})(\partial^{\mu} A^{\nu} - \partial^{\mu} A^{\nu})  - j^{\mu} A^{\mu}\\
    &= -\frac{1}{4}(\partial_{\mu} A_{\nu}(\partial^{\mu} A^{\nu} - \partial^{\nu} A^{\mu})  - \partial_{\nu} A_{\mu}(\partial^{\mu} A^{\nu} - \partial^{\nu} A^{\mu})) - j^{\mu} A^{\mu}\\
    &= -\frac{1}{2}(\partial_{\mu} A_{\nu}(\partial^{\mu} A^{\nu} - \partial^{\nu} A^{\mu})) - j^{\mu} A^{\mu}
\end{align*}

(a)
\begin{align*}
    \frac{\partial \mathcal{L}}{\partial A_{\nu}} &= - j^{\nu} \\
    \frac{\partial \mathcal{L}}{\partial (\partial _{\mu} A_{\nu})} &= -\frac{1}{2}(\partial^{\mu} A^{\nu} - \partial^{\nu} A^{\mu})) -\frac{1}{2}\partial^{\mu} A^{\nu} + \frac{1}{2}\partial^{\nu} A^{\mu} \\ 
    &= -(\partial^{\mu} A^{\nu} - \partial^{\nu} A^{\mu})) = F^{\mu \nu}
\end{align*}
Therefore Euler-Langrangian is 
\begin{align*}
    \frac{\partial \mathcal{L}}{\partial A_{\nu}} - \frac{\partial \mathcal{L}}{\partial (\partial _{\mu} A_{\nu})} &= -j^{\nu} + \partial_{\mu} F^{\mu}{\nu} = 0 \\ 
    \partial_{\mu} F^{\mu \nu} &= j^{\nu}  
\end{align*}

(b) 
taking $\mu = i$ and $\nu = 0$

\begin{align*}
    \partial _{i} F^{i 0} &= j^{0} \\ 
    \implies \overrightarrow{\nabla} \cdot \overrightarrow{E} = \rho
\end{align*}

And taking $\nu = j$
\begin{align*}
    \partial _{\mu} F^{\mu j} &= j^{0} \\
    \implies \partial_{0} F^{0 j} + \partial_{i} F^{ij} &= 0 \\
    \implies \partial^{0} F_{0 j} + \partial^{i} F_{ij} &= 0 \\ \\
    \implies \frac{\partial E_{j}}{\partial t} + \partial ^{i} (-\epsilon_{ijk}B_{k}) &= 0 \\
    \implies \frac{\partial E_{j}}{\partial t} &= \partial ^{i} (-\epsilon_{ijk}B_{k}) \\
    \implies \frac{\partial \boldsymbol{E}}{\partial t} &= \boldsymbol{\nabla} \times \boldsymbol{B} 
\end{align*}

\section*{Solve for problem no. 2}

(a)
Given 
\begin{align*}
    &\partial_{\alpha} F_{\beta \gamma} + \partial_{\beta} F_{\gamma \alpha} + \partial_{\gamma} F_{\alpha \beta} \\
    &= \partial_{\alpha}(\partial_{\beta} A_{\gamma} - \partial_{\gamma} A_{\beta}) + \partial_{\beta}(\partial_{\gamma} A_{\alpha} - \partial_{\alpha} A_{\gamma}) + \partial_{\gamma}(\partial_{\alpha} A_{\beta} - \partial_{\beta} A_{\alpha}) \\ 
    &= \partial_{\alpha} \partial_{\beta}A_{\gamma} - \partial_{\alpha} \partial_{\gamma}A_{\beta} + \partial_{\beta} \partial_{\gamma}A_{\alpha} - \partial_{\beta} \partial_{\alpha}A_{\gamma} + \partial_{\gamma} \partial_{\alpha}A_{\beta} - \partial_{\gamma} \partial_{\beta}A_{\alpha} \\ 
    &= 0
\end{align*}

Given 
\begin{equation}
    \partial_{\alpha} F_{\beta \gamma} + \partial_{\beta} F_{\gamma \alpha} + \partial_{\gamma} F_{\alpha \beta} = 0
\end{equation}

taking $\alpha=1, \beta=2, \gamma=3$ 

\begin{align*}
    \partial_{\alpha} F_{\beta \gamma} + \partial_{\beta} F_{\gamma \alpha} + \partial_{\gamma} F_{\alpha \beta} &= 0\\
    \implies \partial_{1} F_{23} + \partial_{2} F_{31} + \partial_{3} F_{12} &= 0\\
    \implies \partial^{1} B^{1} + \partial^{2} B^{2} + \partial^{3} B^{3}  &= 0\\
    \implies \boldsymbol{\nabla} \cdot \boldsymbol{B} &= 0
\end{align*}

Now taking $\alpha=0, \beta=i, \gamma=j (i\neq j)$
\begin{align*}
    \partial_{\alpha} F_{\beta \gamma} + \partial_{\beta} F_{\gamma \alpha} + \partial_{\gamma} F_{\alpha \beta} &= 0\\
    \implies \partial^{0} F^{ij} + \partial^{j} F^{0i} + \partial^{i} F^{j0} &= 0\\
    \implies -\frac{\partial \boldsymbol{B}}{\partial t} + \partial^{j} E^{i} - \partial^{i} E^{j} &= 0\\
    \implies -\frac{\partial \boldsymbol{B}}{\partial t} &= \epsilon_{ijk} E^{k} \\
    \implies -\frac{\partial \boldsymbol{B}}{\partial t} &= \boldsymbol{\nabla} \times \boldsymbol{E} \\
\end{align*}

\section*{Solve for problem no. 2}

\begin{align*}
    \Pi  _{\mu} &= \frac{\partial \mathcal{L}}{\partial \dot{A_{\mu}} } = \frac{\partial \mathcal{L}}{\partial (\partial _{0}A_{\mu})}  \\
    \implies \frac{\partial \mathcal{L}}{\partial (\partial _{0}A_{\mu})} = - F^{0\mu} \\
    \implies \Pi ^{0} &= - F^{00} = 0 \\
    \implies A_{0} (x) &= 0 \\
\end{align*}
\end{document}