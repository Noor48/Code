\documentclass[12pt, letterpaper]{article}
\usepackage{bbold}
\usepackage{indentfirst}
\usepackage{amsmath, amssymb}
\usepackage[T1]{fontenc}
\usepackage[utf8]{inputenc}
\usepackage{physics}
\usepackage{tensor}
\usepackage{braket}
\usepackage{graphics}
\usepackage{grffile}
\usepackage[export]{adjustbox}
\usepackage{svg}
\usepackage{caption}
\usepackage{subcaption}
\usepackage{authblk} 
\usepackage{setspace}
\newcommand*{\1}{\hspace{1pt}}

\begin{document}
    

    \bibliographystyle{unsrt}
    \section{Out of Time Order Correlator of $H=xp$ model}

    The Riemann hypothesis states that non-trivial zeros of the classical zeta function have real part equal to 1/2. The classical zeta function defined by

    \begin{equation}
        \zeta (s) = \sum_{n = 1}^{\infty} n^{s} 
    \end{equation}

    for Re $s > 1$. 
    By the fundamental theorem of arithmatic, which is also equivalent to the Euler product over primes

    \begin{equation}
        \zeta (s) = \prod _{p} (1-p) ^{-1}
    \end{equation}

    where p are all the prime numbers.

    Zeros of Riemann zeta function are two different types. Trivial zeros of zeta / Riemann zeta function occurs at all negetive integers (for $s = -2, -4, -6, .......$). 
    For complex s (=$\sigma + it $) (with real part between zero and one) , zeta fucntion becomes nontrivial ones. And the Riemann hypothesis is for $s=\frac{1}{2}-iE$
    zeta funtion becomes zero $\zeta(\frac{1}{2}-it) = 0$. Hilbert-P$\acute{o}$lya conjecture suggests that the imaginary parts of the nontrivial zeros are the eiogenvalues 
    of a self-adjoint hamiltonian operator $\hat{H}$. It is also one of the aprroach to proving the Riemann hypothesis. Berry-Keating conjectured that the hamiltonian operator
    of the Hilbert-P$\acute{o}$lya  cinjecture should take the form\cite{s1} 

    \begin{equation}
        \hat{H} _{BK} = \frac{1}{2}(\hat{x}\hat{p} + \hat{p}\hat{x})
    \end{equation}

    Here x and p are position and momentum operators. This 1d classical Hamiltonian ($H = xp$) related to the Riemann zeros.\cite{r1} Berry proposed the Quantum Chaos conjecture, according to 
    which the Riemann zeros are the spectrum of a Hamiltonian obstained by quantization of a classical chaotic hamiltonian, whose periodic orbits are labeled by the prime numbers.
    Connes took the adelic approach to introduce $H = xp $ \cite{s2}. He showed that using different semiclassical regularization, Riemann zeros appear as missing spectral lines in a continuum.

    Now we look into the Berry-Keating and Connes semiclassical approaches to $H=xp$

    \section{Semiclassical approach}

    The classical Berry-Keating-Connes (BKC) Hamiltonian is\cite{s1,s2}

    \begin{equation}
        H ^{cl} _{0} = xp
    \end{equation}

    which has hyperbolic trajectories 
    
    \begin{equation}
        x(t) = x_{0}e^{t}  \ \ \ \  p(t) = p_{0}e^{-t}
    \end{equation}

    So the dynamics is unbounded. There is a continuous spectrum as the quantum level. Berry-Keating and Connes introduced two different types of reularizations and counted
    the semiclassical states. Berry-Keating introduced Plank cell in a phase space: $|x| > l_{x}$ and $|p| > l_{p}$, with $l_{x}l_{p} = 2 \pi \hbar$. Connes 
    choosed $|x| < \Lambda $ and $|p| < \Lambda$, where $\Lambda$ is a cutoff. German Sierra introduced us a third regularization, $l_{x} < x < \Lambda$ combines
    the Berry-Keating and Connes regularization position, not taking assumptions for the momenta p. \

    Semiclassical states number $\mathcal {N}(E)$ with an enery between 0 to E is given by
    \begin{equation}
        \begin{split}
            \mathcal {N}(E) &= \frac{A}{2 \pi \hbar} \\
            &= \frac{A}{h}
        \end{split}
    \end{equation}

    Where A is the area of the allowed phase space region below the curve $E = xp$.
    So the the number of semiclassical states will be for Berry-Keating regularization 
    \begin{equation}
        \begin{split}
            \mathcal {N}_{BK}(E) &= \frac{1}{h}\int_{l_{x}}^{\frac{E}{l_{p}}}  \,dx \int_{l_{p}}^{\frac{E}{x}}  \,dp + ....... \\
            &= \frac{1}{h}\left[\int_{l_{x}}^{\frac{E}{l_{p}}}  \,dx \left[\frac{E}{x} - l_{p}\right] \right]  \\
            &= \frac{1}{h}\left[E\left[\ln x\right] ^{\frac{E}{l_{p}}} _{l_{x}} - l_{p}\left[\frac{E}{l_{p}} - l_{x}\right] \right]  \\
            &= \frac{1}{h}\left[E\ln \frac{E}{l_{x}l_{p}}  - E - l_{x}l_{p} \right]  \\
            &= \frac{1}{h}\left[E\ln \frac{E}{l_{x}l_{p}}  - E - h \right]  \\
            &= \frac{E}{h}\left[\ln \frac{E}{l_{x}l_{p}}  - 1 \right]  + 1 \\
            &= \frac{E}{2 \pi \hbar}\left[\ln \frac{E}{2 \pi \hbar}  - 1 \right]  + 1 \\
        \end{split}
    \end{equation}

    adding Maslov phase $(-\frac{1}{8})$ and $\hbar = 1$, it becomes 
    \begin{equation}
        \mathcal {N}_{BK}(E) = \frac{E}{2 \pi}\left[\ln \frac{E}{2\pi}  - 1 \right]  + \frac{7}{8} + .......,  \ \ \ \ \ \ E>>1\\
    \end{equation}

    The exact formula for the Riemann zeros, $\mathcal{N}_{R} (E)$ contains a fluctuation term which depends on the zeta funcion.\cite{s3}  
    \begin{equation}
        \begin{split}
            \mathcal{N}_{R} (E) &= \left\langle \mathcal{N}\right\rangle + \mathcal{N}_{fl} (E)
            \left\langle\mathcal{N} \right\rangle
        \end{split}
    \end{equation}
    \bibliographystyle{plain}
    \bibliography{ref.bib}
\end{document}