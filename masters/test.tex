\documentclass[12pt, letterpaper]{article}
\usepackage{bbold}
\usepackage{dsfont}
\usepackage{indentfirst}
\usepackage{amsmath, amssymb}
\usepackage[T1]{fontenc}
\usepackage[utf8]{inputenc}
\usepackage{physics}
\usepackage{tensor}
\usepackage{braket}
\usepackage{graphics}
\usepackage{grffile}
\usepackage[export]{adjustbox}
\usepackage{svg}
\usepackage{caption}
\usepackage{subcaption}
\usepackage{authblk} 
\usepackage{setspace}
\usepackage{longtable}
\newcommand*{\1}{\hspace{1pt}}

\begin{document}
    

    \bibliographystyle{unsrt}
    \section{Out of Time Order Correlator of $H=xp$ model}

    The Riemann hypothesis states that non-trivial zeros of the classical zeta function have real part equal to 1/2. The classical zeta function defined by

    \begin{equation}
        \zeta (s) = \sum_{n = 1}^{\infty} n^{s} 
    \end{equation}

    for Re $s > 1$. 
    By the fundamental theorem of arithmatic, which is also equivalent to the Euler product over primes

    \begin{equation}
        \zeta (s) = \prod _{p} (1-p) ^{-1}
    \end{equation}

    where p are all the prime numbers.

    Zeros of Riemann zeta function are two different types. Trivial zeros of zeta / Riemann zeta function occurs at all negetive integers (for $s = -2, -4, -6, .......$). 
    For complex s (=$\sigma + it $) (with real part between zero and one) , zeta fucntion becomes nontrivial ones. And the Riemann hypothesis is for $s=\frac{1}{2}-iE$
    zeta funtion becomes zero $\zeta(\frac{1}{2}-it) = 0$. Hilbert-P$\acute{o}$lya conjecture suggests that the imaginary parts of the nontrivial zeros are the eiogenvalues 
    of a self-adjoint hamiltonian operator $\hat{H}$. It is also one of the aprroach to proving the Riemann hypothesis. Berry-Keating conjectured that the hamiltonian operator
    of the Hilbert-P$\acute{o}$lya  cinjecture should take the form\cite{s1} 

    \begin{equation}
        \hat{H} _{BK} = \frac{1}{2}(\hat{x}\hat{p} + \hat{p}\hat{x})
    \end{equation}

    Here x and p are position and momentum operators. This 1d classical Hamiltonian ($H = xp$) related to the Riemann zeros.\cite{s1} Berry proposed the Quantum Chaos conjecture, according to 
    which the Riemann zeros are the spectrum of a Hamiltonian obstained by quantization of a classical chaotic hamiltonian, whose periodic orbits are labeled by the prime numbers.
    Connes took the adelic approach to introduce $H = xp $ \cite{s2}. He showed that using different semiclassical regularization, Riemann zeros appear as missing spectral lines in a continuum.

    Now we look into the Berry-Keating and Connes semiclassical approaches to $H=xp$

    \section{Semiclassical approach}

    The classical Berry-Keating-Connes (BKC) Hamiltonian is\cite{s1,s2}

    \begin{equation}
        H ^{cl} _{0} = xp
    \end{equation}

    which has hyperbolic trajectories 
    
    \begin{equation}
        x(t) = x_{0}e^{t}  \ \ \ \  p(t) = p_{0}e^{-t}
    \end{equation}

    So the dynamics is unbounded. There is a continuous spectrum as the quantum level. Berry-Keating and Connes introduced two different types of reularizations and counted
    the semiclassical states. Berry-Keating introduced Plank cell in a phase space: $|x| > l_{x}$ and $|p| > l_{p}$, with $l_{x}l_{p} = 2 \pi \hbar$. Connes 
    choosed $|x| < \Lambda $ and $|p| < \Lambda$, where $\Lambda$ is a cutoff. German Sierra introduced us a third regularization, $l_{x} < x < \Lambda$ combines
    the Berry-Keating and Connes regularization position, not taking assumptions for the momenta p. \

    Semiclassical states number $\mathcal {N}(E)$ with an enery between 0 to E is given by
    \begin{equation}
        \begin{split}
            \mathcal {N}(E) &= \frac{A}{2 \pi \hbar} \\
            &= \frac{A}{h}
        \end{split}
    \end{equation}

    Where A is the area of the allowed phase space region below the curve $E = xp$.
    So the the number of semiclassical states will be for Berry-Keating regularization 
    \begin{equation}
        \begin{split}
            \mathcal {N}_{BK}(E) &= \frac{1}{h}\int_{l_{x}}^{\frac{E}{l_{p}}}  \,dx \int_{l_{p}}^{\frac{E}{x}}  \,dp + ....... \\
            &= \frac{1}{h}\left[\int_{l_{x}}^{\frac{E}{l_{p}}}  \,dx \left[\frac{E}{x} - l_{p}\right] \right]  \\
            &= \frac{1}{h}\left[E\left[\ln x\right] ^{\frac{E}{l_{p}}} _{l_{x}} - l_{p}\left[\frac{E}{l_{p}} - l_{x}\right] \right]  \\
            &= \frac{1}{h}\left[E\ln \frac{E}{l_{x}l_{p}}  - E - l_{x}l_{p} \right]  \\
            &= \frac{1}{h}\left[E\ln \frac{E}{l_{x}l_{p}}  - E - h \right]  \\
            &= \frac{E}{h}\left[\ln \frac{E}{l_{x}l_{p}}  - 1 \right]  + 1 \\
            &= \frac{E}{2 \pi \hbar}\left[\ln \frac{E}{2 \pi \hbar}  - 1 \right]  + 1 \\
        \end{split}
    \end{equation}

    adding Maslov phase $(-\frac{1}{8})$ and $\hbar = 1$, it becomes 
    \begin{equation}
        \mathcal {N}_{BK}(E) = \frac{E}{2 \pi}\left[\ln \frac{E}{2\pi}  - 1 \right]  + \frac{7}{8} + .......,  \ \ \ \ \ \ E>>1\\
    \end{equation}

    The exact formula for the Riemann zeros, $\mathcal{N}_{R} (E)$ contains a fluctuation term which depends on the zeta funcion.\cite{s3}  
    \begin{equation}
        \begin{split}
            \mathcal{N}_{R} (E) &= \left\langle \mathcal{N}\right\rangle + \mathcal{N}_{fl} (E) \\
            \left\langle\mathcal{N} (E)\right\rangle &= \frac{1}{\pi} Im \ ln \left[\Gamma\frac{1}{2}\left(\frac{1}{2}-iE\right) \right] - \frac{E}{2 \pi} ln \pi + 1  \\
            \mathcal {N} _{fl} (E) &= \frac{1}{\pi} Im \ ln \left[\zeta\left(\frac{1}{2} - iE \right)\right]
        \end{split}
    \end{equation}

    Bery-Keatin took this result and analogies between formulae in Nunber Theory and Quantum Chaos, they pointed the quantization of classical chaotic Hamiltonian give
    rise to the zeros as point like spectra.\cite{s1,s4} Whereas Connes found the number of semicassical states diverges in the limit where the cutoff $\Lambda$ goes to
    infinity, and that therre us a finite size correction given by mins the average position of the Riemann zeros.

    \begin{equation}
        \begin{split}
            \mathcal {N}_{c} (E) &= \frac{1}{h}\left[ 2E - \left(\frac{E}{\Lambda} \right)^{2} + \int_{\frac{E}{\Lambda}}^{\Lambda}  \,dx\int_{\frac{E}{x}}^{\frac{E}{\Lambda}}  \,dp   \right] \\ 
             & = \frac{1}{h}\left[ 2E - \left(\frac{E}{\Lambda} \right)^{2} + \int_{\frac{E}{\Lambda}}^{\Lambda}  \,dx\left[\frac{E}{x} - \frac{E}{\Lambda}\right]   \right] \\ 
             & = \frac{1}{h}\left[ 2E - \left(\frac{E}{\Lambda} \right)^{2} + E\left[ln \ x\right] _{\frac{E}{\Lambda}}^{\Lambda} - \frac{E}{\Lambda}\left[\Lambda - \frac{E}{\Lambda}\right]   \right] \\ 
             & = \frac{1}{h}\left[ 2E - \left(\frac{E}{\Lambda} \right)^{2} + E\left[ln \ \frac{\Lambda ^{2}}{E}\right] - E + \left(\frac{E}{\Lambda}\right)^{2}\right]  \\ 
             & = \frac{1}{h}\left[ E  + E\left[ln \ \frac{\Lambda ^{2}}{E}\right] \right]  \\ 
             & = \frac{1}{h}\left[ E  + E\left[ln \ \frac{\Lambda ^{2}}{E}\frac{2\pi}{2\pi}\right] \right]  \\ 
             & = \frac{E}{h} ln \ \frac{\Lambda ^{2}}{2\pi} - \frac{E}{h} \left[ln \ \frac{E}{2\pi} - 1\right]  \\
             & = \frac{E}{2\pi} ln \ \frac{\Lambda ^{2}}{2\pi} - \frac{E}{2\pi} \left[ln \ \frac{E}{2\pi} - 1\right] \ \ \ \ \ \ \ \ \ \ \ \ \ \ \ \ \   [taking \ \ \hbar=1] \\
        \end{split} 
    \end{equation}

    This result les to the missing spectral interpretation of the Riemann zeros, according to which there is a continuum of eginstates (represented by the term $\frac{E}{\pi}
     ln \ \Lambda$ in $\mathcal {N} (E)$) where states asscoiated with Riemann zeros are missing.

    Finally, in the S-regularization the number of semiclasical states diverges as $\frac{E}{2\pi} \ ln \ \frac{\Lambda}{l_{x}}$ suggesting a continuum spectrum, ike in 
    Connes's approach. But there is no finite size correction to that formula, and cosequently the possible connection to the Riemann zeros is lost.

    
    \section{Quantization of xp and $\frac{1}{xp}$}
    
    
    \subsection{The Hamitoninan $H_{0} = xp$}

        Here we construst a self adjoint operator $H_{0}$ which acts on a Hilbert space $L^{2}(a,b)$ of square integrable function in the interval $(a,b)$. Taking $x\geqslant 0$
        , there are four possible intervals: $a=0,l_{x}$ and $b=\Lambda, \infty $ where $l_{x}$ and $\Lambda$ were introduced (we shall take $l_{x}$ and $\Lambda = N > 1$).
        Berry-Keating defined the quantnum Hamiltonian $H_{0}$ as the normal ordered expresion 

        \begin{equation}
            H_{0} = \frac{1}{2}(xp + px)
        \end{equation}

        where $p = -i\hbar \frac{d}{dx}$. If $x\geqslant 0$, Eq. () is equivalent to

        \begin{equation}
            H_{0} = \sqrt{x} p \sqrt{x} = -i\hbar \sqrt{x} \frac{d}{dx} \sqrt{x}
        \end{equation}

        This is a symmetric operator acting on a certain domain of the Hilert space $L^{2}(a,b)$, By definition, if an operator is symmetric (or Hermitian)\cite{s5}

        \begin{equation}
            \left\langle\psi | H_{0}\phi \right\rangle = \left\langle\psi H_{0} |\phi \right\rangle
        \end{equation}
         or with limit,
        \begin{equation}
            \left\langle\psi | H_{0}\phi \right\rangle - \left\langle\psi H_{0} |\phi \right\rangle = i \hbar \left[a\psi ^{*}(a)\phi(a) - b\phi ^{*}(b)\psi(b)\right] = 0
        \end{equation}
        
        which is satisfied if both $\psi (x)$ and $\phi (x)$ vanish at the points a, b. von Neumann Theorem of deficiancy indices states that, an operator in symmetric if its deficiency
        indices $n_{\pm }$ are equal.\cite{s6}. Deficiency indices (or the defect numbers) of a closable symmetric operator T are cardinal number S

        \begin{equation}
            \begin{split}
                n_{+} &:= d_{\lambda}  = dim \ \mathcal{R} (T-\overline{\lambda}\mathds{1})^{\perp }  \ \ \ \    Im \ \lambda > 0 \\
                n_{-} &:= d_{\lambda}  = dim \ \mathcal{R} (T-\overline{\lambda}\mathds{1})^{\perp }  \ \ \ \    Im \ \lambda < 0 
            \end{split}
        \end{equation}

        If T is densly defined and symmetric, then T is closable, and by formula $\mathcal{N}(T^{*}) = \mathcal{R}(T)^{\perp}$
        \begin{equation}
            \begin{split}
                n_{+} &:= dim \ \mathcal{N} (T^{*}-i\mathds{1})  = dim \ \mathcal{N} (T^{*}-\lambda \mathds{1})  \ \ \ \    Im \ \lambda > 0 \\
                n_{-} &:= dim \ \mathcal{N} (T^{*}+i\mathds{1})  = dim \ \mathcal{N} (T^{*}+\lambda \mathds{1})  \ \ \ \    Im \ \lambda < 0 
            \end{split}
        \end{equation}

        By definition $n_{\pm} (T) = dim \ \mathcal{N}(T^{*} \mp iT )$ 

        Again if T is a symmetric operator, then
        \begin{equation}
            \begin{split}
                K_{+} &= ker \ (i\mathds{1}-T^{*}) = Ran \ (i\mathds{1} - T) ^{\perp} \\
                K_{-} &= ker \ (i \mathds{1} +T^{*}) = Ran \ (-i\mathds{1} + T) ^{\perp} \\ 
            \end{split}
        \end{equation}
        
        $K_{+}$ and $K_{-}$ are called the deficiency subspaces of T, The pair of numbers $n_{+}$, $n_{-}$ given by $n_{+}(T) = dim[K_{+}]$,$n_{-}(T) = dim[K_{-}]$ arre called
        deficiency indices of T.

        von Neumann Theorem for deficiency indices states that if T an closed operator woth deficiency indices $n_{+}$ and $n_{-}$. Then \\ 
        \\
         (1) T is symmetric if and only if $n_{+} = n_{-} = 0$ ann self adjoint if $\mathcal{D}(T)=\mathcal{D}(T^{*})$ \\ 
         \\ 
         (2) T is symmetric adn self adjoint and also has many self adjoint extensions if and only if $n_{+}=n_{-}\neq 0$ and  $\mathcal{D}(T)=\mathcal{D}(T^{*})$.
         There is one-one correspondence between self adjoint extensions of T and unitary maps from $K_{+}$ onto $K_{-}$ \\ 
         \\ 
         (3) If either $n_{+}=0 \neq n_{-}$ or $n_{-}=0 \neq n_{+}$ then T is not symmetric and has no nontrivial self adjoint extension (such operators are called 
         maximal symmetric operator). \\
         \\
        So this indices counts the number of solutions of the equation, which comes from the deficiency spaces for subsystem T

        \begin{equation}
            n_{\pm} = ker \left(-H_{0}^{\dagger} - \mp i \mathds{1}\right)
        \end{equation}

        which leads to find the solution of the equation.
        \begin{equation}
            H_{0} ^{\dagger} \psi_{\pm} = \pm i \hbar \lambda \psi_{\pm}
        \end{equation}

        belonging to the domain og $H_{0}^{\dagger}(\lambda>0)$. If $n=n_{+}=n_{-}>0$, there are infinitely many self-adoint extensions of $H_{0}$ parameterized by a 
        unitary $n\times n$ matrix. Stone's theorem states that if $U(t)$ be a strongly continuous one parameter unitary group on a Hilbert space $\mathcal{H}$. 
        Then, there is a self-adjoint operator A on $\mathcal{H}$ so that $U(t) = e^{itA}$. The solution of the equation () is 

        \begin{align*}
            &H_{0}^{\dagger} \psi_{\pm} = \pm i \hbar \lambda \psi_{\pm} \\ 
            &\implies H_{0} \psi_{\pm} = \pm i \hbar \lambda \psi_{\pm} \ \ \ \ \ [becuase\ H_{0} \ is \ self-adjoint] \\
            &\implies \left(-i\hbar \sqrt{x} \frac{d}{dx}\sqrt{x}\right)\psi_{\pm} = \pm i \hbar \lambda \psi _{\pm} \\
            &\implies -i\hbar \sqrt{x} \frac{d}{dx}\left(\sqrt{x}\psi_{\pm}\right) = \pm i \hbar \lambda \psi _{\pm} \\
            &\implies -\sqrt{x} \frac{d}{dx}\left(\sqrt{x}\psi_{\pm}\right) = \pm \lambda \psi _{\pm} \\
            &\implies -x\frac{d}{dx}\psi_{\pm} - \sqrt{x}\frac{1}{2\sqrt{x}} \frac{d}{dx}\psi_{\pm} = \pm \lambda \psi _{\pm} \\
            &\implies -x\frac{d}{dx}\psi_{\pm} = \left(\pm \lambda + \frac{1}{2}\right) \psi _{\pm} \\
            &\implies \frac{d}{dx}\psi_{\pm} = -\frac{1}{x}\left(\pm \lambda + \frac{1}{2}\right) \psi _{\pm} \\
            &\implies \frac{d\psi_{\pm}}{\psi_{\pm}} = -\frac{dx}{x}\left(\pm \lambda + \frac{1}{2}\right) \\
            &\implies ln \ \psi_{\pm} = -\left(ln \ x\right)\left(\pm \lambda + \frac{1}{2}\right) + ln \ C\\
            &\implies \psi_{\pm} =  Cx^{- \frac{1}{2}\mp \lambda}\\
        \end{align*}

        whose norm in the interval (a,b) is
        \begin{equation}
            \begin{split}
            \left\langle \psi_{\pm} | \psi_{\pm} \right\rangle &= \int_{a}^{b} C^{2} x^{-1\mp \lambda} \,dx \\
            &= \mp \frac{C^{2}}{2\lambda}\left(b^{\mp 2\lambda} - a^{\mp 2\lambda}\right) \\  
            &= \pm \frac{C^{2}}{2\lambda}\left(a^{\mp 2\lambda} - b^{\mp 2\lambda}\right) 
            \end{split}
        \end{equation}
        The deficiency indices correponding to the four intervals cosidered above are collecte in Table. We find the deficency indices by observing different intervals.
        For BK intervals$(1,\infty)$ only $\psi_{+}$ belongs to hilbert space ($\psi_{-}$ blows out. or putting intervals in Eq. (20) and testing whether it belongs to
        the Hilbert space). And the rest are given below
        
        \begin{longtable}[c]{c c c c c}
            \caption{Deficiency indices of $H_{0}$. The corresponding intervals are associated to the semiclassical regularizations of section 2 (BK, C ,S).
            The last one T, describes the case with no constraints on x except positivity (i.e. x>0)} \\
            \hline
             Type & (a,b) & $(n_{+}, n_{-})$ & Self-adjoint\\
             \hline 
             BK & $(1,\infty)$ & (1,0) & -\\  
             C & (0,N) & (0,1) &  -\\ 
             S & (1,N) & (1,1) & $\surd$ \\ 
             T & $(0,\infty)$ & (0,0) & $\surd $\\ 
            \hline
        \end{longtable}


        From the von Neumann theorem we see that $H_{0}$
    \bibliographystyle{plain}
    \bibliography{ref.bib}
\end{document}