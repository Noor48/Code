\documentclass[12pt, letterpaper]{article}
\usepackage{indentfirst}
\usepackage{amsmath, amssymb}
\usepackage[T1]{fontenc}
\usepackage[utf8]{inputenc}
\usepackage{physics}
\usepackage{tensor}
\usepackage{braket}
\usepackage{graphics}
\usepackage{grffile}
\usepackage[export]{adjustbox}
\usepackage{svg}
\usepackage{caption}
\usepackage{subcaption}
\usepackage{authblk} 
\usepackage{setspace}

\begin{document}
    \section{Out of Time Order Correlator of $H=xp$ model}

    The Riemann hypothesis states that non-trivial zeros of the classical zeta function have real part equal to 1/2. The classical zeta function defined by

    \begin{equation}
        \zeta (s) = \sum_{n = 1}^{\infty} n^{s} 
    \end{equation}

    for Re $s > 1$. 
    By the fundamental theorem of arithmatic, which is also equivalent to the Euler product over primes

    \begin{equation}
        \zeta (s) = \prod _{p} (1-p) ^{-1}
    \end{equation}

    where p are all the prime numbers.

    Zeros of Riemann zeta function are two different types. Trivial zeros of zeta / Riemann zeta function occurs at all negetive integers (for $s = -2, -4, -6, .......$). 
    For complex s (=$\sigma + it $) (with real part between zero and one) , zeta fucntion becomes nontrivial ones. And the Riemann hypothesis is for $s=\frac{1}{2}-iE$
    zeta funtion becomes zero $\zeta(\frac{1}{2}-iE) = 0$. Hilbert-P$\acute{o}$lya

    
\end{document}