\documentclass[12pt, letterpaper]{article}
\usepackage{amsmath}
\usepackage{tensor}
\usepackage{stix}
\usepackage{slashed}
\usepackage{hyperref}
\usepackage[compat=1.0.0]{tikz-feynman}
\usepackage{feynmp-auto}
\usepackage{float}
\usepackage{braket}
\usepackage{graphicx} % Required for inserting images
%\usepackage[paperheight=6in,
%   paperwidth=5in,
%   top=10mm,
%   bottom=20mm,
%   left=10mm,
%   right=10mm]{geometry}
\newcommand*{\1}{\hspace{1pt}}
\title{Solution to Problem Sheet 5}
\author{Noor E Mustafa Ferdous\\ email: \href{mailto:nooremf@gmail.com}{nooremf@gmail.com} }

\date{}

\begin{document}

\maketitle

\section*{Solve for problem no. 1}

Given
\begin{align*}
    \int \prod ^{N} _{i=1} dx^{i} exp\bigg(-\frac{1}{2}x^{T}Ax + J^{T}x\bigg) = \frac{(2\pi)^{N/2}}{\sqrt{det \ A}}exp\bigg(\frac{1}{2}J^{T}A^{-1}J\bigg)
\end{align*}

L.H.S 
\begin{align*}
    \int \prod ^{N} _{i=1} dx^{i} exp\bigg(-\frac{1}{2}x^{T}Ax + J^{T}x\bigg)
\end{align*}

where
\begin{align*}
    x &= (x_{1}, .....  , x_{n}) \  \epsilon \ \mathbb{R} ^{n} \\
    J &= (J_{1}, .....  , J_{n}) \  \epsilon \ \mathbb{C} ^{n} \\
    A &= 
    \begin{pmatrix}
        A_{11} & A_{12} &. &. & .\\
        A_{21} & A_{22} & .& .& .\\
        .& .& .& & &\\
        .& .& & .& &\\
        .& .& & & A_{nn}
        \end{pmatrix}
         \ \epsilon  \ \mathbb{R}^{n \times n}
\end{align*}

where A is symmetric and + ve definite.
From L.H.S.
\begin{align*}
    \int \prod ^{N} _{i=1} dx^{i} exp\bigg(-\frac{1}{2}x^{T}Ax + J^{T}x\bigg)
\end{align*}

$x^{T}Ax$ is
\begin{align*}
    \sum^{n}_{i,j=1}x_{i}A_{ij}x_{j}
\end{align*}
which is 
\begin{align*}
    \sum^{n}_{i=1}\sum_{i<j} (x_{i}A_{ij}x_{j} + x_{j}A_{ji}x_{i}) + \sum^{n}_{i=1} A_{ii}x_{i}x_{i}
\end{align*}
Any anti symmetric term will drop out.
Symmetric matrices are diagonalisable. 
that is 
\begin{align*}
    \exists O \in O(n) &= {A \in \mathbb{R} ^{n \times n} | O^{T}O = I_{n}} \\
    D &= diag(\lambda_{1}, ......., \lambda_{n}) \ ~ \lambda_{i} \ are \ the \ eigenvalues \ of \ A
\end{align*}
s.t.
\begin{align*}
    A &= O^{T}DO \\ 
    x^{T}Ax &= x^{t}O^{T}DOx \\ 
\end{align*}
and 
\begin{align*}
    J^{T}x &=  J^{T}O^{T}(Ox) \\
    &= (OJ)^{T}(Ox) \\
\end{align*}

Let 
\begin{align*}
    y &= Ox \\
    y_{i} &= O_{ij}x_{j}
\end{align*}

And
\begin{align*}
    d^{n} y &= \big|\frac{\partial y}{\partial x}\big| d^{n}x \\
    &= \big|det \ \frac{\partial y_{i}}{\partial x_{j}} | \\
    &= |det \ O| \tag{$\frac{\partial y_{i}}{\partial x_{j}}= O_{ij}$}\\
    &= 1
\end{align*}

and 
\begin{align*}
    D_{ij} = \lambda_{i}\delta_{ij}
\end{align*}

Now 
\begin{align*}
&\int _{\mathbb{R}} d^{n}y e^{-\frac{1}{2}y^{T}Dy+J^{\prime}y} \tag{$(OJ)^{T}=J^{\prime}$}\\ 
&= \int _{\mathbb{R}} d^{n}y e^{-\frac{1}{2}\sum_{i}\sim{j}y_{i}\lambda_{i} \delta_{ij} y_{j} +\sum_{i}J^{\prime}_{i}y_{i}} \\
&= \int _{\mathbb{R}} d^{n}y e^{\sum_{i}(-\frac{1}{2}\sim{j}y^{2}_{i}\lambda_{i} +J^{\prime}_{i}y_{i})}\\ 
&= \prod_{i} \int _{\mathbb{R}} d^{n}y e^{-\frac{1}{2}\sim{j}y^{2}_{i}\lambda_{i} +J^{\prime}_{i}y_{i}} \\
&= \prod_{i} \sqrt{\frac{2\pi}{\lambda_{i}}} e^{\frac{1}{2}\frac{J^{\prime}_{i}}{\lambda_{i}}}
\end{align*}

Now 
\begin{align*}
    D^{-1} _{ij} &= \frac{\delta_{ij}}{\lambda_{i}} \\
    \frac{J^{\prime}_{i}}{\lambda_{i}} &= J^{\prime T} D^{-1} J^{\prime} \\ 
    &= J O^{T} D^{-1} O J \\
    &= M^{-1}
\end{align*}

therefore
\begin{align*}
    \int \prod ^{N} _{i=1} dx^{i} exp\bigg(-\frac{1}{2}x^{T}Ax + J^{T}x\bigg) = \sqrt{\frac{(2\pi)^{n}}{det \ M}} e^{\frac{1}{2}J^{T}M^{-1}J}
\end{align*}
\section*{Solve for problem no. 3}
We know
\begin{align*}
    \langle 0 | T\phi(x_{1})\phi(x_{2}) | 0 \rangle &= \frac{1}{Z_{0}}\frac{\partial}{\partial J(x_{1})}\frac{\partial}{\partial J(x_{2})} exp\bigg[-\frac{1}{2}\int d^{4}x d^{4}y J(x)D_{F}(x-y) J(y) \bigg] Z[J] \bigg|_{J=0} \\
\end{align*}
\end{document}